%% Generated by Sphinx.
\def\sphinxdocclass{report}
\documentclass[letterpaper,10pt,english]{sphinxmanual}
\ifdefined\pdfpxdimen
   \let\sphinxpxdimen\pdfpxdimen\else\newdimen\sphinxpxdimen
\fi \sphinxpxdimen=.75bp\relax
\ifdefined\pdfimageresolution
    \pdfimageresolution= \numexpr \dimexpr1in\relax/\sphinxpxdimen\relax
\fi
%% let collapsible pdf bookmarks panel have high depth per default
\PassOptionsToPackage{bookmarksdepth=5}{hyperref}

\PassOptionsToPackage{booktabs}{sphinx}
\PassOptionsToPackage{colorrows}{sphinx}

\PassOptionsToPackage{warn}{textcomp}
\usepackage[utf8]{inputenc}
\ifdefined\DeclareUnicodeCharacter
% support both utf8 and utf8x syntaxes
  \ifdefined\DeclareUnicodeCharacterAsOptional
    \def\sphinxDUC#1{\DeclareUnicodeCharacter{"#1}}
  \else
    \let\sphinxDUC\DeclareUnicodeCharacter
  \fi
  \sphinxDUC{00A0}{\nobreakspace}
  \sphinxDUC{2500}{\sphinxunichar{2500}}
  \sphinxDUC{2502}{\sphinxunichar{2502}}
  \sphinxDUC{2514}{\sphinxunichar{2514}}
  \sphinxDUC{251C}{\sphinxunichar{251C}}
  \sphinxDUC{2572}{\textbackslash}
\fi
\usepackage{cmap}
\usepackage[T1]{fontenc}
\usepackage{amsmath,amssymb,amstext}
\usepackage{babel}



\usepackage{tgtermes}
\usepackage{tgheros}
\renewcommand{\ttdefault}{txtt}



\usepackage[Bjarne]{fncychap}
\usepackage{sphinx}

\fvset{fontsize=auto}
\usepackage{geometry}


% Include hyperref last.
\usepackage{hyperref}
% Fix anchor placement for figures with captions.
\usepackage{hypcap}% it must be loaded after hyperref.
% Set up styles of URL: it should be placed after hyperref.
\urlstyle{same}

\addto\captionsenglish{\renewcommand{\contentsname}{Contents:}}

\usepackage{sphinxmessages}
\setcounter{tocdepth}{1}



\title{pyCATCH}
\date{Nov 09, 2023}
\release{}
\author{Stephan G.\@{} Heinemann}
\newcommand{\sphinxlogo}{\vbox{}}
\renewcommand{\releasename}{}
\makeindex
\begin{document}

\ifdefined\shorthandoff
  \ifnum\catcode`\=\string=\active\shorthandoff{=}\fi
  \ifnum\catcode`\"=\active\shorthandoff{"}\fi
\fi

\pagestyle{empty}
\sphinxmaketitle
\pagestyle{plain}
\sphinxtableofcontents
\pagestyle{normal}
\phantomsection\label{\detokenize{index::doc}}


\noindent\sphinxincludegraphics{{catch}.jpg}

\sphinxAtStartPar
Welcome to pyCATCH: the phython implementation of the Collection of Analysis Tools for Coronal Holes (CATCH; Heinemann et al. 2019)

\sphinxAtStartPar
CATCH was originally implemented in SSW IDL (See \sphinxurl{http://lmsal.com/solarsoft/ssw\_install.html} and \sphinxurl{http://www.lmsal.com/solarsoft/ssw\_packages\_info.html}).
It is also available on GitHub (\sphinxurl{https://github.com/sgheinemann/CATCH})

\sphinxAtStartPar
CATCH and pyCATCH were created in order to collect and structure coronal hole identification, extraction and analysis in a handy and fast way without the disadvantages of automatic algorithms. It enables the user to download and process EUV filtergrams (193/195 A) and line\sphinxhyphen{}of\sphinxhyphen{}sight (Los) magnetograms. It is able to handle data from different spacecraft missions covering the interval from 1996 until now. These include the Solar Dynamics Observatory, the Solar Terrestrial Relations Observatory and the Solar and Heliospheric Observatory.

\sphinxAtStartPar
The user can perform coronal hole boundary detection, extraction and analysis using a manually adjustable intensity threshold. Additionally the user can analyze the underlying photospheric magnetic field.

\sphinxAtStartPar
If you have any comments, suggestions or need help. Please contact the author via E\sphinxhyphen{}mail (\sphinxhref{mailto:stephan.heinemann@hmail.at}{stephan.heinemann@hmail.at}) or leave a request on Github (\sphinxurl{https://github.com/sgheinemann/CATCH} or \sphinxurl{https://github.com/sgheinemann/pycatch} respectively).

\sphinxAtStartPar
You can read the documentation by opening the

\begin{sphinxVerbatim}[commandchars=\\\{\}]
\PYG{n}{doc}\PYG{o}{/}\PYG{n}{\PYGZus{}build}\PYG{o}{/}\PYG{n}{index}\PYG{o}{.}\PYG{n}{html}
\end{sphinxVerbatim}

\sphinxAtStartPar
file.

\sphinxstepscope


\chapter{Getting started with pyCATCH}
\label{\detokenize{getting_started:getting-started-with-pycatch}}\label{\detokenize{getting_started::doc}}
\sphinxAtStartPar
This section provides a quick introduction to using pyCATCH.

\sphinxAtStartPar
pyCATCH essentially consists only of the class object pycatch that inclused inbuilt methods for detection and extraction of coronal holes.

\sphinxAtStartPar
It uses a lot of sunpy \textless{}\sphinxurl{https://sunpy.org/}\textgreater{} functionalites, thus for advanced usage of pyCATCH, it is suggested to make yourself familiar with it.


\section{Reading the documentation}
\label{\detokenize{getting_started:reading-the-documentation}}
\sphinxAtStartPar
You can read the documentation by opening the

\begin{sphinxVerbatim}[commandchars=\\\{\}]
\PYG{n}{doc}\PYG{o}{/}\PYG{n}{\PYGZus{}build}\PYG{o}{/}\PYG{n}{html}\PYG{o}{/}\PYG{n}{index}\PYG{o}{.}\PYG{n}{html}
\end{sphinxVerbatim}

\sphinxAtStartPar
file.


\section{Installing pyCATCH}
\label{\detokenize{getting_started:installing-pycatch}}
\sphinxAtStartPar
For pyCATCH version \textless{} 1.0.0:    Download or pull pyCATCH repository from GitHub

\begin{sphinxVerbatim}[commandchars=\\\{\}]
\PYG{n}{https}\PYG{p}{:}\PYG{o}{/}\PYG{o}{/}\PYG{n}{github}\PYG{o}{.}\PYG{n}{com}\PYG{o}{/}\PYG{n}{sgheinemann}\PYG{o}{/}\PYG{n}{pycatch}
\end{sphinxVerbatim}

\sphinxAtStartPar
Navigate to the directory and install pyCATCH using pip

\begin{sphinxVerbatim}[commandchars=\\\{\}]
\PYG{n}{pip} \PYG{n}{install} \PYG{o}{.}
\end{sphinxVerbatim}

\sphinxAtStartPar
pyCATCH uses the following packages:
\begin{itemize}
\item {} 
\sphinxAtStartPar
aiapy

\item {} 
\sphinxAtStartPar
astropy

\item {} 
\sphinxAtStartPar
joblib

\item {} 
\sphinxAtStartPar
matplotlib

\item {} 
\sphinxAtStartPar
numexpr

\item {} 
\sphinxAtStartPar
numpy

\item {} 
\sphinxAtStartPar
opencv\sphinxhyphen{}python

\item {} 
\sphinxAtStartPar
sunpy

\item {} 
\sphinxAtStartPar
reproject

\item {} 
\sphinxAtStartPar
scipy

\end{itemize}


\section{Initializing pyCATCH}
\label{\detokenize{getting_started:initializing-pycatch}}
\sphinxAtStartPar
Import the pycatch class with

\begin{sphinxVerbatim}[commandchars=\\\{\}]
\PYG{k+kn}{from} \PYG{n+nn}{pycatch}\PYG{n+nn}{.}\PYG{n+nn}{pycatch} \PYG{k+kn}{import} \PYG{n}{pycatch}
\end{sphinxVerbatim}

\sphinxAtStartPar
then you can initilize the class

\begin{sphinxVerbatim}[commandchars=\\\{\}]
\PYG{n}{ch} \PYG{o}{=} \PYG{n}{pycatch}\PYG{p}{(}\PYG{p}{)}
\end{sphinxVerbatim}


\section{Dowloading data}
\label{\detokenize{getting_started:dowloading-data}}
\sphinxAtStartPar
EUV data can be downloaded with

\begin{sphinxVerbatim}[commandchars=\\\{\}]
\PYG{n}{ch}\PYG{o}{.}\PYG{n}{download}\PYG{p}{(}\PYG{l+s+s1}{\PYGZsq{}}\PYG{l+s+s1}{DATE}\PYG{l+s+s1}{\PYGZsq{}}\PYG{p}{)}
\end{sphinxVerbatim}

\sphinxAtStartPar
and respectively magnetograms, if a EUV map is already loaded

\begin{sphinxVerbatim}[commandchars=\\\{\}]
\PYG{n}{ch}\PYG{o}{.}\PYG{n}{download\PYGZus{}magnetogram}\PYG{p}{(}\PYG{p}{)}
\end{sphinxVerbatim}


\section{Loading data}
\label{\detokenize{getting_started:loading-data}}
\sphinxAtStartPar
EUV images and magnetograms can be loaded into the pyCATCH class by

\begin{sphinxVerbatim}[commandchars=\\\{\}]
\PYG{n}{ch}\PYG{o}{.}\PYG{n}{load}\PYG{p}{(}\PYG{n}{file}\PYG{o}{=}\PYG{l+s+s1}{\PYGZsq{}}\PYG{l+s+s1}{PATH}\PYG{l+s+s1}{\PYGZsq{}}\PYG{p}{)}
\PYG{n}{ch}\PYG{o}{.}\PYG{n}{load}\PYG{p}{(}\PYG{n}{file}\PYG{o}{=}\PYG{l+s+s1}{\PYGZsq{}}\PYG{l+s+s1}{PATH}\PYG{l+s+s1}{\PYGZsq{}}\PYG{p}{,} \PYG{n}{mag} \PYG{o}{=} \PYG{k+kc}{True} \PYG{p}{)}
\end{sphinxVerbatim}

\sphinxAtStartPar
If data was downloaded with the same instance of the pyCATCH class, the path to the downloaded data is stored in the class and the load method can be called without providing the path.


\section{Calibrating data}
\label{\detokenize{getting_started:calibrating-data}}
\sphinxAtStartPar
Prepare the date for coronal hole extraction

\begin{sphinxVerbatim}[commandchars=\\\{\}]
\PYG{n}{ch}\PYG{o}{.}\PYG{n}{calibration}\PYG{p}{(}\PYG{p}{)}
\PYG{n}{ch}\PYG{o}{.}\PYG{n}{calibration\PYGZus{}mag}\PYG{p}{(}\PYG{p}{)}
\end{sphinxVerbatim}

\sphinxAtStartPar
The data can rebinned and cutout if desired

\begin{sphinxVerbatim}[commandchars=\\\{\}]
\PYG{n}{ch}\PYG{o}{.}\PYG{n}{rebin}\PYG{p}{(}\PYG{n}{ndim}\PYG{o}{=}\PYG{p}{(}\PYG{n}{nx}\PYG{p}{,}\PYG{n}{ny}\PYG{p}{)}\PYG{p}{)}
\PYG{n}{ch}\PYG{o}{.}\PYG{n}{cutout}\PYG{p}{(}\PYG{n}{top}\PYG{o}{=}\PYG{p}{(}\PYG{n}{x\PYGZus{}max}\PYG{p}{,}\PYG{n}{y\PYGZus{}max}\PYG{p}{)}\PYG{p}{,} \PYG{n}{bot}\PYG{o}{=}\PYG{p}{(}\PYG{n}{x\PYGZus{}min}\PYG{p}{,}\PYG{n}{y\PYGZus{}min}\PYG{p}{)}\PYG{p}{)}
\end{sphinxVerbatim}


\section{Selecting coronal hole seed point}
\label{\detokenize{getting_started:selecting-coronal-hole-seed-point}}
\sphinxAtStartPar
Select a seed point from an interactive window from where the coronal hole will be grown

\begin{sphinxVerbatim}[commandchars=\\\{\}]
\PYG{n}{ch}\PYG{o}{.}\PYG{n}{select}\PYG{p}{(}\PYG{n}{hint}\PYG{o}{=}\PYG{k+kc}{True}\PYG{p}{)}
\end{sphinxVerbatim}

\sphinxAtStartPar
Flag hint to highlight dark regions (possible but not necessarily all coronal hole).


\section{Setting a threshold}
\label{\detokenize{getting_started:setting-a-threshold}}
\sphinxAtStartPar
pyCATCH features four different options to select a threshold.
\begin{itemize}
\item {} 
\sphinxAtStartPar
Set the threshold manually

\begin{sphinxVerbatim}[commandchars=\\\{\}]
\PYG{n}{ch}\PYG{o}{.}\PYG{n}{set\PYGZus{}threshold}\PYG{p}{(}\PYG{n}{Threshold}\PYG{p}{)}
\end{sphinxVerbatim}

\item {} 
\sphinxAtStartPar
Use the threshold derived from CATCH statistics (Heinemann et al. 2019)

\begin{sphinxVerbatim}[commandchars=\\\{\}]
\PYG{n}{ch}\PYG{o}{.}\PYG{n}{suggest\PYGZus{}threshold}\PYG{p}{(}\PYG{p}{)}
\end{sphinxVerbatim}

\sphinxAtStartPar
It is advised to use this option to get a starting suggestion and then adjust the threshold as needed.

\item {} 
\sphinxAtStartPar
Select the threshold from solar disk intensity distribution

\begin{sphinxVerbatim}[commandchars=\\\{\}]
\PYG{n}{ch}\PYG{o}{.}\PYG{n}{threshold\PYGZus{}from\PYGZus{}hist}\PYG{p}{(}\PYG{p}{)}
\end{sphinxVerbatim}

\item {} 
\sphinxAtStartPar
Calculate the coronal hole area and uncertainty as function of intensity and select the boundary to be where the uncertainty is lowest.

\begin{sphinxVerbatim}[commandchars=\\\{\}]
\PYG{n}{ch}\PYG{o}{.}\PYG{n}{calculate\PYGZus{}curves}\PYG{p}{(}\PYG{p}{)}
\PYG{n}{ch}\PYG{o}{.}\PYG{n}{threshold\PYGZus{}from\PYGZus{}curves}\PYG{p}{(}\PYG{p}{)}
\end{sphinxVerbatim}

\sphinxAtStartPar
Warning: This is an advance option and can be slow with high resolution.

\end{itemize}


\section{Extracting the coronal hole}
\label{\detokenize{getting_started:extracting-the-coronal-hole}}
\sphinxAtStartPar
The coronal hole can be extracted from the selected seed point and the set threshold

\begin{sphinxVerbatim}[commandchars=\\\{\}]
\PYG{n}{ch}\PYG{o}{.}\PYG{n}{extract\PYGZus{}ch}\PYG{p}{(}\PYG{p}{)}
\end{sphinxVerbatim}


\section{Calculating coronal hole properties}
\label{\detokenize{getting_started:calculating-coronal-hole-properties}}
\sphinxAtStartPar
The coronal hole’s morphological properties can be calculated with

\begin{sphinxVerbatim}[commandchars=\\\{\}]
\PYG{n}{ch}\PYG{o}{.}\PYG{n}{calculate\PYGZus{}properties}\PYG{p}{(}\PYG{p}{)}
\end{sphinxVerbatim}

\sphinxAtStartPar
and the magnetic properties can be calculated with

\begin{sphinxVerbatim}[commandchars=\\\{\}]
\PYG{n}{ch}\PYG{o}{.}\PYG{n}{calculate\PYGZus{}properties}\PYG{p}{(}\PYG{n}{mag}\PYG{o}{=}\PYG{k+kc}{True}\PYG{p}{)}
\end{sphinxVerbatim}


\section{Plotting and saving coronal hole extractions}
\label{\detokenize{getting_started:plotting-and-saving-coronal-hole-extractions}}
\sphinxAtStartPar
The intensity map with the coronal hole boundary and uncertainties overlaid can be displayed with

\begin{sphinxVerbatim}[commandchars=\\\{\}]
\PYG{n}{ch}\PYG{o}{.}\PYG{n}{plot\PYGZus{}map}\PYG{p}{(}\PYG{p}{)}
\end{sphinxVerbatim}

\sphinxAtStartPar
Set the ‘mag’ flag to display the magnetogram instead

\begin{sphinxVerbatim}[commandchars=\\\{\}]
\PYG{n}{ch}\PYG{o}{.}\PYG{n}{plot\PYGZus{}map}\PYG{p}{(}\PYG{n}{mag}\PYG{o}{=}\PYG{k+kc}{True}\PYG{p}{)}
\end{sphinxVerbatim}

\sphinxAtStartPar
Set the ‘save’ flag to save the images to PDF

\begin{sphinxVerbatim}[commandchars=\\\{\}]
\PYG{n}{ch}\PYG{o}{.}\PYG{n}{plot\PYGZus{}map}\PYG{p}{(}\PYG{n}{save}\PYG{o}{=}\PYG{k+kc}{True}\PYG{p}{)}
\PYG{n}{ch}\PYG{o}{.}\PYG{n}{plot\PYGZus{}map}\PYG{p}{(}\PYG{n}{mag}\PYG{o}{=}\PYG{k+kc}{True}\PYG{p}{,}\PYG{n}{save}\PYG{o}{=}\PYG{k+kc}{True}\PYG{p}{)}
\end{sphinxVerbatim}

\sphinxAtStartPar
The properties of the extracted coronal hole can be saved in a text file with

\begin{sphinxVerbatim}[commandchars=\\\{\}]
\PYG{n}{ch}\PYG{o}{.}\PYG{n}{print\PYGZus{}properties}\PYG{p}{(}\PYG{p}{)}
\end{sphinxVerbatim}

\sphinxAtStartPar
And the extracted coronal hole ‘binary’ map can be saved as a fits file with

\begin{sphinxVerbatim}[commandchars=\\\{\}]
\PYG{n}{ch}\PYG{o}{.}\PYG{n}{bin2fits}\PYG{p}{(}\PYG{p}{)}
\end{sphinxVerbatim}


\section{Saving and loading pyCATCH session}
\label{\detokenize{getting_started:saving-and-loading-pycatch-session}}
\sphinxAtStartPar
A pyCATCH object can be save to a pickle file

\begin{sphinxVerbatim}[commandchars=\\\{\}]
\PYG{n}{ch}\PYG{o}{.}\PYG{n}{save}\PYG{p}{(}\PYG{p}{)}
\end{sphinxVerbatim}

\sphinxAtStartPar
and restored with

\begin{sphinxVerbatim}[commandchars=\\\{\}]
\PYG{n}{ch\PYGZus{}loaded} \PYG{o}{=} \PYG{n}{pycatch}\PYG{p}{(}\PYG{n}{restore}\PYG{o}{=}\PYG{l+s+s1}{\PYGZsq{}}\PYG{l+s+s1}{PATH}\PYG{l+s+s1}{\PYGZsq{}}\PYG{p}{)}
\end{sphinxVerbatim}


\section{Example}
\label{\detokenize{getting_started:example}}
\sphinxAtStartPar
Example python scripts can be found in the directory

\begin{sphinxVerbatim}[commandchars=\\\{\}]
\PYG{n}{examples}\PYG{o}{/}
\end{sphinxVerbatim}

\sphinxstepscope


\chapter{pyCATCH}
\label{\detokenize{pycatch/index:pycatch}}\label{\detokenize{pycatch/index::doc}}
\sphinxstepscope


\section{pycatch.pycatch}
\label{\detokenize{pycatch/pycatch:pycatch-pycatch}}\label{\detokenize{pycatch/pycatch::doc}}
\sphinxAtStartPar
The \sphinxtitleref{pycatch} class is the primary access point to pyCATCH’s functionalities.
\index{pycatch (class in pycatch.pycatch)@\spxentry{pycatch}\spxextra{class in pycatch.pycatch}}

\begin{fulllineitems}
\phantomsection\label{\detokenize{pycatch/pycatch:pycatch.pycatch.pycatch}}
\pysigstartsignatures
\pysiglinewithargsret{\sphinxbfcode{\sphinxupquote{class\DUrole{w}{  }}}\sphinxcode{\sphinxupquote{pycatch.pycatch.}}\sphinxbfcode{\sphinxupquote{pycatch}}}{\emph{\DUrole{n}{restore}\DUrole{o}{=}\DUrole{default_value}{None}}, \emph{\DUrole{o}{**}\DUrole{n}{kwargs}}}{}
\pysigstopsignatures
\sphinxAtStartPar
Bases: \sphinxcode{\sphinxupquote{object}}

\sphinxAtStartPar
A Python library for extracting and analyzing coronal holes from solar EUV images and magnetograms.


\subsection{Attributes}
\label{\detokenize{pycatch/pycatch:attributes}}\begin{quote}
\begin{description}
\sphinxlineitem{dir}{[}str{]}
\sphinxAtStartPar
The directory path. Defaults to the user’s home directory if not provided.

\sphinxlineitem{save\_dir}{[}str{]}
\sphinxAtStartPar
The save directory path. Defaults to the user’s home directory if not provided.

\sphinxlineitem{map\_file}{[}str{]}
\sphinxAtStartPar
The map file path.

\sphinxlineitem{magnetogram\_file}{[}str{]}
\sphinxAtStartPar
The magnetogram file path.

\sphinxlineitem{map}{[}sunpy.map.Map{]}
\sphinxAtStartPar
The loaded and configured EUV map.
This map contains both the 2D data array and metadata associated with the EUV observation.

\sphinxlineitem{original\_map}{[}sunpy.map.Map{]}
\sphinxAtStartPar
The original EUV map.
This map contains the initially loaded EUV observation before any operations.

\sphinxlineitem{magnetogram}{[}sunpy.map.Map{]}
\sphinxAtStartPar
The loaded and configured magnetogram.
This map contains both the 2D data array and metadata associated with the magnetic field data.

\sphinxlineitem{point}{[}list of float{]}
\sphinxAtStartPar
Seed point for coronal hole extraction.

\sphinxlineitem{curves}{[}tuple{]}\begin{itemize}
\item {} 
\sphinxAtStartPar
The threshold range.

\item {} 
\sphinxAtStartPar
The calculated area curves for the coronal hole.

\item {} 
\sphinxAtStartPar
The uncertainty in the area curves.

\end{itemize}

\sphinxlineitem{threshold}{[}float{]}
\sphinxAtStartPar
Coronal hole extraction threshold.

\sphinxlineitem{type}{[}str{]}
\sphinxAtStartPar
Placeholder for type information.

\sphinxlineitem{rebin\_status}{[}bool{]}
\sphinxAtStartPar
Whether the map was rebinned.

\sphinxlineitem{cutout\_status}{[}bool{]}
\sphinxAtStartPar
Whether the map was cutout.

\sphinxlineitem{kernel}{[}int{]}
\sphinxAtStartPar
Size of the circular kernel for morphological operations.

\sphinxlineitem{binmap}{[}sunpy.map.Map{]}
\sphinxAtStartPar
Single 5\sphinxhyphen{}level binary map with the coronal hole extraction, where each level represents a different threshold value.

\sphinxlineitem{properties}{[}dict{]}
\sphinxAtStartPar
A dictionary containing the calculate coronal hole properties.

\sphinxlineitem{\_\_version\_\_}{[}str{]}
\sphinxAtStartPar
Version number of pyCATCH

\end{description}
\end{quote}


\subsection{Parameters}
\label{\detokenize{pycatch/pycatch:parameters}}\begin{description}
\sphinxlineitem{dir}{[}str, optional{]}
\sphinxAtStartPar
Directory for storing and loading data. Default is the home directory.

\sphinxlineitem{save\_dir}{[}str, optional{]}
\sphinxAtStartPar
Directory for storing data. Default is the home directory.

\sphinxlineitem{map\_file}{[}str, optional{]}
\sphinxAtStartPar
Filepath to EUV/Intensity map, needs to be loadable with sunpy.map.Map(). Default is None.

\sphinxlineitem{magnetogram\_file}{[}str, optional{]}
\sphinxAtStartPar
Filepath to magnetogram, needs to be loadable with sunpy.map.Map(). Default is None.

\sphinxlineitem{load}{[}str, optional{]}
\sphinxAtStartPar
Loads a previously saved pyCATCH object from the specified path, which overrides any other keywords. Default is None.

\end{description}


\subsection{Returns}
\label{\detokenize{pycatch/pycatch:returns}}
\sphinxAtStartPar
None


\subsection{Methods}
\label{\detokenize{pycatch/pycatch:methods}}\index{bin2fits() (pycatch.pycatch.pycatch method)@\spxentry{bin2fits()}\spxextra{pycatch.pycatch.pycatch method}}

\begin{fulllineitems}
\phantomsection\label{\detokenize{pycatch/pycatch:pycatch.pycatch.pycatch.bin2fits}}
\pysigstartsignatures
\pysiglinewithargsret{\sphinxbfcode{\sphinxupquote{bin2fits}}}{\emph{\DUrole{n}{file}\DUrole{o}{=}\DUrole{default_value}{None}}, \emph{\DUrole{n}{small}\DUrole{o}{=}\DUrole{default_value}{False}}, \emph{\DUrole{n}{overwrite}\DUrole{o}{=}\DUrole{default_value}{False}}}{}
\pysigstopsignatures
\sphinxAtStartPar
Save the coronal hole binary map to a FITS file.


\subsubsection{Parameters}
\label{\detokenize{pycatch/pycatch:id1}}\begin{description}
\sphinxlineitem{file}{[}str, optional{]}
\sphinxAtStartPar
Filepath to save the FITS file. If not provided, a default filename will be generated based on observation metadata. Default is None.

\sphinxlineitem{small}{[}bool, optional{]}
\sphinxAtStartPar
Save a smaller region around the coronal hole. Default is False.

\sphinxlineitem{overwrite}{[}bool, optional{]}
\sphinxAtStartPar
Flag to overwrite the file if it already exists. Default is False.

\end{description}


\subsubsection{Returns}
\label{\detokenize{pycatch/pycatch:id2}}
\sphinxAtStartPar
None

\end{fulllineitems}

\index{calculate\_curves() (pycatch.pycatch.pycatch method)@\spxentry{calculate\_curves()}\spxextra{pycatch.pycatch.pycatch method}}

\begin{fulllineitems}
\phantomsection\label{\detokenize{pycatch/pycatch:pycatch.pycatch.pycatch.calculate_curves}}
\pysigstartsignatures
\pysiglinewithargsret{\sphinxbfcode{\sphinxupquote{calculate\_curves}}}{\emph{\DUrole{n}{verbose}\DUrole{o}{=}\DUrole{default_value}{True}}}{}
\pysigstopsignatures
\sphinxAtStartPar
Calculate area and uncertainty curves as a function of intensity.


\subsubsection{Parameters}
\label{\detokenize{pycatch/pycatch:id3}}\begin{quote}
\begin{description}
\sphinxlineitem{verbose}{[}bool, optional{]}
\sphinxAtStartPar
Display warnings. Default is True.

\end{description}
\end{quote}


\subsubsection{Returns}
\label{\detokenize{pycatch/pycatch:id4}}
\sphinxAtStartPar
None

\end{fulllineitems}

\index{calculate\_properties() (pycatch.pycatch.pycatch method)@\spxentry{calculate\_properties()}\spxextra{pycatch.pycatch.pycatch method}}

\begin{fulllineitems}
\phantomsection\label{\detokenize{pycatch/pycatch:pycatch.pycatch.pycatch.calculate_properties}}
\pysigstartsignatures
\pysiglinewithargsret{\sphinxbfcode{\sphinxupquote{calculate\_properties}}}{\emph{\DUrole{n}{mag}\DUrole{o}{=}\DUrole{default_value}{False}}, \emph{\DUrole{n}{align}\DUrole{o}{=}\DUrole{default_value}{False}}}{}
\pysigstopsignatures
\sphinxAtStartPar
Calculate the morphological coronal hole properties from the extracted binary map.


\subsubsection{Parameters}
\label{\detokenize{pycatch/pycatch:id5}}\begin{quote}
\begin{description}
\sphinxlineitem{mag}{[}bool, optional{]}
\sphinxAtStartPar
Calculate magnetic properties instead. Default is False.

\sphinxlineitem{align}{[}bool, optional{]}
\sphinxAtStartPar
Call pycatch.calibration\_mag() to align with binary map. Default is False.

\end{description}
\end{quote}


\subsubsection{Returns}
\label{\detokenize{pycatch/pycatch:id6}}
\sphinxAtStartPar
None

\end{fulllineitems}

\index{calibration() (pycatch.pycatch.pycatch method)@\spxentry{calibration()}\spxextra{pycatch.pycatch.pycatch method}}

\begin{fulllineitems}
\phantomsection\label{\detokenize{pycatch/pycatch:pycatch.pycatch.pycatch.calibration}}
\pysigstartsignatures
\pysiglinewithargsret{\sphinxbfcode{\sphinxupquote{calibration}}}{\emph{\DUrole{o}{**}\DUrole{n}{kwargs}}}{}
\pysigstopsignatures
\sphinxAtStartPar
Calibrate the intensity image.


\subsubsection{Parameters}
\label{\detokenize{pycatch/pycatch:id7}}\begin{quote}
\begin{description}
\sphinxlineitem{** kwargs (SDO/AIA) :}\begin{description}
\sphinxlineitem{deconvolve}{[}bool or numpy.ndarray, optional{]}
\sphinxAtStartPar
Use PSF deconvolution. Default is None. It takes a custom PSF array as input, if True uses aiapy.psf.deconvolve.
WARNING: can take about 10 minutes.

\sphinxlineitem{register}{[}bool, optional{]}
\sphinxAtStartPar
Co\sphinxhyphen{}register the map. Default is True.

\sphinxlineitem{normalize}{[}bool, optional{]}
\sphinxAtStartPar
Normalize intensity to 1s. Default is True.

\sphinxlineitem{degradation}{[}bool, optional{]}
\sphinxAtStartPar
Correct instrument degradation. Default is True.

\sphinxlineitem{alc}{[}bool, optional{]}
\sphinxAtStartPar
Apply Annulus Limb Correction, Python implementation from Verbeek et al. (2014). Default is True.

\sphinxlineitem{cut\_limb}{[}bool, optional{]}
\sphinxAtStartPar
Set off\sphinxhyphen{}limb pixel values to NaN. Default is True.

\end{description}

\sphinxlineitem{** kwargs (STEREO/SECCHI) :}\begin{description}
\sphinxlineitem{deconvolve}{[}bool, optional{]}
\sphinxAtStartPar
NOT YET IMPLEMENTED FOR STEREO.

\sphinxlineitem{register}{[}bool, optional{]}
\sphinxAtStartPar
Co\sphinxhyphen{}register the map. Default is True.

\sphinxlineitem{normalize}{[}bool, optional{]}
\sphinxAtStartPar
Normalize intensity to 1s. Default is True.

\sphinxlineitem{alc}{[}bool, optional{]}
\sphinxAtStartPar
Apply Annulus Limb Correction, Python implementation from Verbeek et al. (2014). Default is True.

\sphinxlineitem{cut\_limb}{[}bool, optional{]}
\sphinxAtStartPar
Set off\sphinxhyphen{}limb pixel values to NaN. Default is True.

\end{description}

\end{description}
\end{quote}


\subsubsection{Returns}
\label{\detokenize{pycatch/pycatch:id8}}
\sphinxAtStartPar
None

\end{fulllineitems}

\index{calibration\_mag() (pycatch.pycatch.pycatch method)@\spxentry{calibration\_mag()}\spxextra{pycatch.pycatch.pycatch method}}

\begin{fulllineitems}
\phantomsection\label{\detokenize{pycatch/pycatch:pycatch.pycatch.pycatch.calibration_mag}}
\pysigstartsignatures
\pysiglinewithargsret{\sphinxbfcode{\sphinxupquote{calibration\_mag}}}{\emph{\DUrole{o}{**}\DUrole{n}{kwargs}}}{}
\pysigstopsignatures
\sphinxAtStartPar
Calibrate the magnetogram.


\subsubsection{Parameters}
\label{\detokenize{pycatch/pycatch:id9}}\begin{quote}
\begin{description}
\sphinxlineitem{** kwargs (SDO/HMI)}{[}{]}\begin{description}
\sphinxlineitem{rotate}{[}bool, optional{]}
\sphinxAtStartPar
Rotate the map so that North is up. Default is True.

\sphinxlineitem{align}{[}bool, optional{]}
\sphinxAtStartPar
Align with an AIA map. Default is True.

\sphinxlineitem{cut\_limb}{[}bool, optional{]}
\sphinxAtStartPar
Set off\sphinxhyphen{}limb pixel values to NaN. Default is True.

\end{description}

\end{description}
\end{quote}


\subsubsection{Returns}
\label{\detokenize{pycatch/pycatch:id10}}
\sphinxAtStartPar
None

\end{fulllineitems}

\index{cutout() (pycatch.pycatch.pycatch method)@\spxentry{cutout()}\spxextra{pycatch.pycatch.pycatch method}}

\begin{fulllineitems}
\phantomsection\label{\detokenize{pycatch/pycatch:pycatch.pycatch.pycatch.cutout}}
\pysigstartsignatures
\pysiglinewithargsret{\sphinxbfcode{\sphinxupquote{cutout}}}{\emph{\DUrole{n}{top}\DUrole{o}{=}\DUrole{default_value}{(1100, 1100)}}, \emph{\DUrole{n}{bot}\DUrole{o}{=}\DUrole{default_value}{(\sphinxhyphen{}1100, \sphinxhyphen{}1100)}}}{}
\pysigstopsignatures
\sphinxAtStartPar
Cut a subfield of the map (if a magnetogram is loaded, it will also be cut).


\subsubsection{Parameters}
\label{\detokenize{pycatch/pycatch:id11}}\begin{quote}
\begin{description}
\sphinxlineitem{top}{[}tuple, optional{]}
\sphinxAtStartPar
Coordinates of the top\sphinxhyphen{}right corner. Default is (1100, 1100).

\sphinxlineitem{bot}{[}tuple, optional{]}
\sphinxAtStartPar
Coordinates of the bottom\sphinxhyphen{}left corner. Default is (\sphinxhyphen{}1100, \sphinxhyphen{}1100)).

\end{description}
\end{quote}


\subsubsection{Returns}
\label{\detokenize{pycatch/pycatch:id12}}
\sphinxAtStartPar
None

\end{fulllineitems}

\index{download() (pycatch.pycatch.pycatch method)@\spxentry{download()}\spxextra{pycatch.pycatch.pycatch method}}

\begin{fulllineitems}
\phantomsection\label{\detokenize{pycatch/pycatch:pycatch.pycatch.pycatch.download}}
\pysigstartsignatures
\pysiglinewithargsret{\sphinxbfcode{\sphinxupquote{download}}}{\emph{\DUrole{n}{time}}, \emph{\DUrole{n}{instr}\DUrole{o}{=}\DUrole{default_value}{\textquotesingle{}AIA\textquotesingle{}}}, \emph{\DUrole{n}{wave}\DUrole{o}{=}\DUrole{default_value}{193}}, \emph{\DUrole{n}{source}\DUrole{o}{=}\DUrole{default_value}{\textquotesingle{}SDO\textquotesingle{}}}, \emph{\DUrole{o}{**}\DUrole{n}{kwargs}}}{}
\pysigstopsignatures
\sphinxAtStartPar
Download an EUV map using VSO (Virtual Solar Observatory). It downloads the closest image within +/\sphinxhyphen{} 1 hour of the time provided.


\subsubsection{Parameters}
\label{\detokenize{pycatch/pycatch:id13}}\begin{quote}
\begin{description}
\sphinxlineitem{time}{[}tuple, list, str, pandas.Timestamp, pandas.Series, pandas.DatetimeIndex, datetime.datetime, datetime.date, numpy.datetime64, numpy.ndarray, astropy.time.Time{]}
\sphinxAtStartPar
The time of the EUV image to download. Input needs be parsed by sunpy.time.parse\_time()

\sphinxlineitem{instrument}{[}str, optional{]}
\sphinxAtStartPar
The instrument of the EUV image to download. Default is ‘AIA’.

\sphinxlineitem{source}{[}str, optional{]}
\sphinxAtStartPar
The source of the EUV image to download. Default is ‘SDO’.

\sphinxlineitem{wavelength}{[}int, optional{]}
\sphinxAtStartPar
The wavelength of the EUV image to download (in Angstrom). Default is 193.

\sphinxlineitem{** kwargs: }
\sphinxAtStartPar
Additional keyword arguments passed to sunpy.Fido.search (see sunpy documentation for more information).

\end{description}
\end{quote}


\subsubsection{Returns}
\label{\detokenize{pycatch/pycatch:id14}}
\sphinxAtStartPar
None


\subsubsection{Notes}
\label{\detokenize{pycatch/pycatch:notes}}
\sphinxAtStartPar
For EIT data: instr=’EIT’, wave=195, source=’SOHO’
For STEREO\sphinxhyphen{}A data: instr=’SECCHI’, wave=195, source=’STEREO\_A’
For STEREO\sphinxhyphen{}B data: instr=’SECCHI’, wave=195, source=’STEREO\_B’

\end{fulllineitems}

\index{download\_magnetogram() (pycatch.pycatch.pycatch method)@\spxentry{download\_magnetogram()}\spxextra{pycatch.pycatch.pycatch method}}

\begin{fulllineitems}
\phantomsection\label{\detokenize{pycatch/pycatch:pycatch.pycatch.pycatch.download_magnetogram}}
\pysigstartsignatures
\pysiglinewithargsret{\sphinxbfcode{\sphinxupquote{download\_magnetogram}}}{\emph{\DUrole{n}{cadence}\DUrole{o}{=}\DUrole{default_value}{45}}, \emph{\DUrole{n}{time}\DUrole{o}{=}\DUrole{default_value}{None}}, \emph{\DUrole{o}{**}\DUrole{n}{kwargs}}}{}
\pysigstopsignatures
\sphinxAtStartPar
Download a magnetogram matching the EUV image date using VSO (Virtual Solar Observatory). It downloads the closest image within +/\sphinxhyphen{} 1 hour ot the time of the EUV image.


\subsubsection{Parameters}
\label{\detokenize{pycatch/pycatch:id15}}\begin{quote}
\begin{description}
\sphinxlineitem{cadence}{[}int, optional{]}
\sphinxAtStartPar
Download Line\sphinxhyphen{}of\sphinxhyphen{}Sight (LOS) magnetogram with the specified cadence in seconds. Default is 45.

\sphinxlineitem{time}{[}tuple, None or list, str, pandas.Timestamp, pandas.Series, pandas.DatetimeIndex, datetime.datetime, datetime.date, numpy.datetime64, numpy.ndarray, astropy.time.Time{]}
\sphinxAtStartPar
The time of the magnetogram to download. Input needs be parsed by sunpy.time.parse\_time(), optional
Overrides the date of the EUV map and the filepath of the downloaded data is not stored in self.magnetogram\_file.

\sphinxlineitem{** kwargs}{[}{]}
\sphinxAtStartPar
Additional keyword arguments passed to sunpy.Fido.search (see sunpy documentation for more information).

\end{description}
\end{quote}


\subsubsection{Returns}
\label{\detokenize{pycatch/pycatch:id16}}
\sphinxAtStartPar
None

\end{fulllineitems}

\index{extract\_ch() (pycatch.pycatch.pycatch method)@\spxentry{extract\_ch()}\spxextra{pycatch.pycatch.pycatch method}}

\begin{fulllineitems}
\phantomsection\label{\detokenize{pycatch/pycatch:pycatch.pycatch.pycatch.extract_ch}}
\pysigstartsignatures
\pysiglinewithargsret{\sphinxbfcode{\sphinxupquote{extract\_ch}}}{\emph{\DUrole{n}{kernel}\DUrole{o}{=}\DUrole{default_value}{None}}}{}
\pysigstopsignatures
\sphinxAtStartPar
Extract the coronal hole from the intensity map using the selected threshold and seed point.
This function outputs a binary map to pycatch.binmap.


\subsubsection{Parameters}
\label{\detokenize{pycatch/pycatch:id17}}\begin{quote}
\begin{description}
\sphinxlineitem{kernel}{[}int or None, optional{]}
\sphinxAtStartPar
Size of the circular kernel for morphological operations. Default is None. If None, a kernel size depending on resolution will be used.

\end{description}
\end{quote}


\subsubsection{Returns}
\label{\detokenize{pycatch/pycatch:id18}}
\sphinxAtStartPar
None

\end{fulllineitems}

\index{load() (pycatch.pycatch.pycatch method)@\spxentry{load()}\spxextra{pycatch.pycatch.pycatch method}}

\begin{fulllineitems}
\phantomsection\label{\detokenize{pycatch/pycatch:pycatch.pycatch.pycatch.load}}
\pysigstartsignatures
\pysiglinewithargsret{\sphinxbfcode{\sphinxupquote{load}}}{\emph{\DUrole{n}{file}\DUrole{o}{=}\DUrole{default_value}{None}}, \emph{\DUrole{n}{mag}\DUrole{o}{=}\DUrole{default_value}{False}}}{}
\pysigstopsignatures
\sphinxAtStartPar
Load maps.


\subsubsection{Parameters}
\label{\detokenize{pycatch/pycatch:id19}}\begin{quote}
\begin{description}
\sphinxlineitem{mag}{[}bool, optional{]}
\sphinxAtStartPar
Flag to load a magnetogram. Default is False.

\sphinxlineitem{file}{[}str, optional{]}
\sphinxAtStartPar
Filepath to load a specific map. If not set, it loads pycatch.map\_file or pycatch.magnetogram\_file. Default is None.

\end{description}
\end{quote}


\subsubsection{Returns}
\label{\detokenize{pycatch/pycatch:id20}}
\sphinxAtStartPar
None

\end{fulllineitems}

\index{plot\_map() (pycatch.pycatch.pycatch method)@\spxentry{plot\_map()}\spxextra{pycatch.pycatch.pycatch method}}

\begin{fulllineitems}
\phantomsection\label{\detokenize{pycatch/pycatch:pycatch.pycatch.pycatch.plot_map}}
\pysigstartsignatures
\pysiglinewithargsret{\sphinxbfcode{\sphinxupquote{plot\_map}}}{\emph{\DUrole{n}{boundary}\DUrole{o}{=}\DUrole{default_value}{True}}, \emph{\DUrole{n}{uncertainty}\DUrole{o}{=}\DUrole{default_value}{True}}, \emph{\DUrole{n}{original}\DUrole{o}{=}\DUrole{default_value}{False}}, \emph{\DUrole{n}{small}\DUrole{o}{=}\DUrole{default_value}{False}}, \emph{\DUrole{n}{cutout}\DUrole{o}{=}\DUrole{default_value}{None}}, \emph{\DUrole{n}{grid}\DUrole{o}{=}\DUrole{default_value}{False}}, \emph{\DUrole{n}{mag}\DUrole{o}{=}\DUrole{default_value}{False}}, \emph{\DUrole{n}{fsize}\DUrole{o}{=}\DUrole{default_value}{(10, 10)}}, \emph{\DUrole{n}{save}\DUrole{o}{=}\DUrole{default_value}{False}}, \emph{\DUrole{n}{sfile}\DUrole{o}{=}\DUrole{default_value}{None}}, \emph{\DUrole{n}{overwrite}\DUrole{o}{=}\DUrole{default_value}{False}}, \emph{\DUrole{o}{**}\DUrole{n}{kwargs}}}{}
\pysigstopsignatures
\sphinxAtStartPar
Display a coronal hole plot.


\subsubsection{Parameters}
\label{\detokenize{pycatch/pycatch:id21}}\begin{quote}
\begin{description}
\sphinxlineitem{boundary}{[}bool, optional{]}
\sphinxAtStartPar
Overplot the coronal hole boundary. Default is True.

\sphinxlineitem{uncertainty}{[}bool, optional{]}
\sphinxAtStartPar
Show the uncertainty of the coronal hole boundary. Default is True.

\sphinxlineitem{original}{[}bool, optional{]}
\sphinxAtStartPar
Show the original image. Default is False.

\sphinxlineitem{small}{[}bool, optional{]}
\sphinxAtStartPar
Plot a smaller region around the coronal hole. Default is False.
Overrides cutout.

\sphinxlineitem{cutout}{[}list of tuple, optional{]}
\sphinxAtStartPar
Display a cutout around the extracted coronal hole. Format: {[}(xbot, ybot), (xtop, ytop){]}. Default is None.

\sphinxlineitem{grid}{[}bool, optional{]}
\sphinxAtStartPar
Display grid. Default if False.

\sphinxlineitem{mag}{[}bool, optional{]}
\sphinxAtStartPar
Show a magnetogram instead of the coronal hole plot. Default is False.

\sphinxlineitem{fsize}{[}tuple, optional{]}
\sphinxAtStartPar
Set the figure size (in inch). Default is (10, 10).

\sphinxlineitem{save}{[}bool, optional{]}
\sphinxAtStartPar
Save and close the figure. Default is False.

\sphinxlineitem{sfile}{[}str, optional{]}
\sphinxAtStartPar
Filepath to save the image as pdf. If not provided, a default filename will be generated based on observation metadata.
Use only in conjunction with save=True. Default is None.

\sphinxlineitem{overwrite}{[}bool, optional{]}
\sphinxAtStartPar
Overwrite the plot if it already exists. Default is False.

\sphinxlineitem{** kwargs}{[}keyword arguments{]}
\sphinxAtStartPar
Additional keyword arguments for sunpy.map.Map.plot(). See the sunpy documentation for more information.

\end{description}
\end{quote}


\subsubsection{Returns}
\label{\detokenize{pycatch/pycatch:id22}}
\sphinxAtStartPar
None

\end{fulllineitems}

\index{print\_properties() (pycatch.pycatch.pycatch method)@\spxentry{print\_properties()}\spxextra{pycatch.pycatch.pycatch method}}

\begin{fulllineitems}
\phantomsection\label{\detokenize{pycatch/pycatch:pycatch.pycatch.pycatch.print_properties}}
\pysigstartsignatures
\pysiglinewithargsret{\sphinxbfcode{\sphinxupquote{print\_properties}}}{\emph{\DUrole{n}{file}\DUrole{o}{=}\DUrole{default_value}{None}}, \emph{\DUrole{n}{overwrite}\DUrole{o}{=}\DUrole{default_value}{False}}}{}
\pysigstopsignatures
\sphinxAtStartPar
Save properties to a text file.


\subsubsection{Parameters}
\label{\detokenize{pycatch/pycatch:id23}}\begin{quote}
\begin{description}
\sphinxlineitem{file}{[}str, optional{]}
\sphinxAtStartPar
Filepath to save the data. Default is pycatch.dir.

\sphinxlineitem{overwrite}{[}bool, optional{]}
\sphinxAtStartPar
Flag to overwrite the file if it already exists. Default is False.

\end{description}
\end{quote}


\subsubsection{Returns}
\label{\detokenize{pycatch/pycatch:id24}}
\sphinxAtStartPar
None

\end{fulllineitems}

\index{rebin() (pycatch.pycatch.pycatch method)@\spxentry{rebin()}\spxextra{pycatch.pycatch.pycatch method}}

\begin{fulllineitems}
\phantomsection\label{\detokenize{pycatch/pycatch:pycatch.pycatch.pycatch.rebin}}
\pysigstartsignatures
\pysiglinewithargsret{\sphinxbfcode{\sphinxupquote{rebin}}}{\emph{\DUrole{n}{ndim}\DUrole{o}{=}\DUrole{default_value}{(1024, 1024)}}, \emph{\DUrole{o}{**}\DUrole{n}{kwargs}}}{}
\pysigstopsignatures
\sphinxAtStartPar
Rebin maps to a new resolution (if a magnetogram is loaded, it will also be resampled).


\subsubsection{Parameters}
\label{\detokenize{pycatch/pycatch:id25}}\begin{quote}
\begin{description}
\sphinxlineitem{ndim}{[}tuple, optional{]}
\sphinxAtStartPar
New dimensions of the map. Default is (1024, 1024)).

\sphinxlineitem{** kwargs}{[}{]}
\sphinxAtStartPar
Additional keyword arguments passed to sunpy.map.Map.resample (see sunpy documentation for more information).

\end{description}
\end{quote}


\subsubsection{Returns}
\label{\detokenize{pycatch/pycatch:id26}}
\sphinxAtStartPar
None

\end{fulllineitems}

\index{save() (pycatch.pycatch.pycatch method)@\spxentry{save()}\spxextra{pycatch.pycatch.pycatch method}}

\begin{fulllineitems}
\phantomsection\label{\detokenize{pycatch/pycatch:pycatch.pycatch.pycatch.save}}
\pysigstartsignatures
\pysiglinewithargsret{\sphinxbfcode{\sphinxupquote{save}}}{\emph{\DUrole{n}{file}\DUrole{o}{=}\DUrole{default_value}{None}}, \emph{\DUrole{n}{overwrite}\DUrole{o}{=}\DUrole{default_value}{False}}, \emph{\DUrole{n}{no\_original}\DUrole{o}{=}\DUrole{default_value}{True}}}{}
\pysigstopsignatures
\sphinxAtStartPar
Save a pyCATCH object to a pickle file.


\subsubsection{Parameters}
\label{\detokenize{pycatch/pycatch:id27}}\begin{quote}
\begin{description}
\sphinxlineitem{file}{[}str, optional{]}
\sphinxAtStartPar
Filepath to save the object, default is pycatch.dir. Default is None.

\sphinxlineitem{overwrite}{[}bool, optional{]}
\sphinxAtStartPar
Flag to overwrite the file if it already exists. Default is False.

\sphinxlineitem{no\_original}{[}bool, optional{]}
\sphinxAtStartPar
Flag to exclude saving the original map to save disk space. Default is True.

\end{description}
\end{quote}


\subsubsection{Returns}
\label{\detokenize{pycatch/pycatch:id28}}
\sphinxAtStartPar
None

\end{fulllineitems}

\index{select() (pycatch.pycatch.pycatch method)@\spxentry{select()}\spxextra{pycatch.pycatch.pycatch method}}

\begin{fulllineitems}
\phantomsection\label{\detokenize{pycatch/pycatch:pycatch.pycatch.pycatch.select}}
\pysigstartsignatures
\pysiglinewithargsret{\sphinxbfcode{\sphinxupquote{select}}}{\emph{\DUrole{n}{hint}\DUrole{o}{=}\DUrole{default_value}{False}}, \emph{\DUrole{n}{fsize}\DUrole{o}{=}\DUrole{default_value}{(10, 10)}}}{}
\pysigstopsignatures
\sphinxAtStartPar
Select a seed point from the intensity map.


\subsubsection{Parameters}
\label{\detokenize{pycatch/pycatch:id29}}\begin{quote}
\begin{description}
\sphinxlineitem{hint}{[}bool, optional{]}
\sphinxAtStartPar
If True, highlights possible coronal holes. Default is False.

\sphinxlineitem{fsize}{[}tuple, optional{]}
\sphinxAtStartPar
Set the figure size (in inch). Default is (10, 10).

\end{description}
\end{quote}


\subsubsection{Returns}
\label{\detokenize{pycatch/pycatch:id30}}
\sphinxAtStartPar
None

\end{fulllineitems}

\index{set\_threshold() (pycatch.pycatch.pycatch method)@\spxentry{set\_threshold()}\spxextra{pycatch.pycatch.pycatch method}}

\begin{fulllineitems}
\phantomsection\label{\detokenize{pycatch/pycatch:pycatch.pycatch.pycatch.set_threshold}}
\pysigstartsignatures
\pysiglinewithargsret{\sphinxbfcode{\sphinxupquote{set\_threshold}}}{\emph{\DUrole{n}{threshold}}, \emph{\DUrole{n}{median}\DUrole{o}{=}\DUrole{default_value}{True}}, \emph{\DUrole{n}{no\_percentage}\DUrole{o}{=}\DUrole{default_value}{False}}}{}
\pysigstopsignatures
\sphinxAtStartPar
Set the coronal hole extraction threshold.


\subsubsection{Parameters}
\label{\detokenize{pycatch/pycatch:id31}}\begin{quote}
\begin{description}
\sphinxlineitem{threshold}{[}float{]}
\sphinxAtStartPar
The threshold value.

\sphinxlineitem{median}{[}bool, optional{]}
\sphinxAtStartPar
If True, the input is assumed to be a fraction of the median solar disk intensity. Default is True.

\sphinxlineitem{no\_percentage}{[}bool, optional{]}
\sphinxAtStartPar
If True, the input is given as a percentage of the median solar disk intensity. Default is False.
This only works in conjunction with median=True.

\end{description}
\end{quote}


\subsubsection{Returns}
\label{\detokenize{pycatch/pycatch:id32}}
\sphinxAtStartPar
None

\end{fulllineitems}

\index{suggest\_threshold() (pycatch.pycatch.pycatch method)@\spxentry{suggest\_threshold()}\spxextra{pycatch.pycatch.pycatch method}}

\begin{fulllineitems}
\phantomsection\label{\detokenize{pycatch/pycatch:pycatch.pycatch.pycatch.suggest_threshold}}
\pysigstartsignatures
\pysiglinewithargsret{\sphinxbfcode{\sphinxupquote{suggest\_threshold}}}{}{}
\pysigstopsignatures
\sphinxAtStartPar
Suggest a coronal hole extraction threshold based on the CATCH statistics (Heinemann et al. 2019).

\sphinxAtStartPar
This function calculates a threshold value using the formula:
TH = 0.29 * Im + 11.53 {[}DN{]}

\sphinxAtStartPar
where Im is the median solar disk intensity in Data Numbers (DN).


\subsubsection{Parameters}
\label{\detokenize{pycatch/pycatch:id33}}
\sphinxAtStartPar
None


\subsubsection{Returns}
\label{\detokenize{pycatch/pycatch:id34}}
\sphinxAtStartPar
None

\end{fulllineitems}

\index{threshold\_from\_curves() (pycatch.pycatch.pycatch method)@\spxentry{threshold\_from\_curves()}\spxextra{pycatch.pycatch.pycatch method}}

\begin{fulllineitems}
\phantomsection\label{\detokenize{pycatch/pycatch:pycatch.pycatch.pycatch.threshold_from_curves}}
\pysigstartsignatures
\pysiglinewithargsret{\sphinxbfcode{\sphinxupquote{threshold\_from\_curves}}}{\emph{\DUrole{n}{fsize}\DUrole{o}{=}\DUrole{default_value}{(10, 5)}}}{}
\pysigstopsignatures
\sphinxAtStartPar
Select a threshold for coronal hole extraction from calculated area and uncertainty curves as a function of intensity.

\sphinxAtStartPar
Before using this function, you need to calculate the curves using \sphinxtitleref{pycatch.calculate\_curves()}.


\subsubsection{Parameters}
\label{\detokenize{pycatch/pycatch:id35}}\begin{quote}
\begin{description}
\sphinxlineitem{fsize}{[}tuple, optional{]}
\sphinxAtStartPar
Set the figure size (in inch). Default is (10, 5).

\end{description}
\end{quote}


\subsubsection{Returns}
\label{\detokenize{pycatch/pycatch:id36}}
\sphinxAtStartPar
None

\end{fulllineitems}

\index{threshold\_from\_hist() (pycatch.pycatch.pycatch method)@\spxentry{threshold\_from\_hist()}\spxextra{pycatch.pycatch.pycatch method}}

\begin{fulllineitems}
\phantomsection\label{\detokenize{pycatch/pycatch:pycatch.pycatch.pycatch.threshold_from_hist}}
\pysigstartsignatures
\pysiglinewithargsret{\sphinxbfcode{\sphinxupquote{threshold\_from\_hist}}}{\emph{\DUrole{n}{fsize}\DUrole{o}{=}\DUrole{default_value}{(10, 5)}}}{}
\pysigstopsignatures
\sphinxAtStartPar
Select a threshold for coronal hole extraction from the solar disk intensity histogram.


\subsubsection{Parameters}
\label{\detokenize{pycatch/pycatch:id37}}\begin{quote}
\begin{description}
\sphinxlineitem{fsize}{[}tuple, optional{]}
\sphinxAtStartPar
Set the figure size (in inch). Default is (10, 5).

\end{description}
\end{quote}


\subsubsection{Returns}
\label{\detokenize{pycatch/pycatch:id38}}
\sphinxAtStartPar
None

\end{fulllineitems}


\end{fulllineitems}


\sphinxstepscope


\section{pycatch.utils}
\label{\detokenize{pycatch/utils/index:pycatch-utils}}\label{\detokenize{pycatch/utils/index::doc}}
\sphinxAtStartPar
The \sphinxtitleref{utils} package contains utility submodules.

\sphinxstepscope


\subsection{pycatch.utils.calibration Module}
\label{\detokenize{pycatch/utils/calibration:pycatch-utils-calibration-module}}\label{\detokenize{pycatch/utils/calibration::doc}}
\sphinxAtStartPar
The \sphinxtitleref{calibration} module provides calibration functions.

\phantomsection\label{\detokenize{pycatch/utils/calibration:module-pycatch.utils.calibration}}\index{module@\spxentry{module}!pycatch.utils.calibration@\spxentry{pycatch.utils.calibration}}\index{pycatch.utils.calibration@\spxentry{pycatch.utils.calibration}!module@\spxentry{module}}\index{annulus\_limb\_correction() (in module pycatch.utils.calibration)@\spxentry{annulus\_limb\_correction()}\spxextra{in module pycatch.utils.calibration}}

\begin{fulllineitems}
\phantomsection\label{\detokenize{pycatch/utils/calibration:pycatch.utils.calibration.annulus_limb_correction}}
\pysigstartsignatures
\pysiglinewithargsret{\sphinxcode{\sphinxupquote{pycatch.utils.calibration.}}\sphinxbfcode{\sphinxupquote{annulus\_limb\_correction}}}{\emph{\DUrole{n}{map}}}{}
\pysigstopsignatures
\sphinxAtStartPar
Apply annulus limb correction to a solar map.

\sphinxAtStartPar
This function performs annulus limb correction on a solar map following the method described in Verbeek et al. (2014).
Transferred from IDL to Python 3 by S.G. Heinemann, June 2022


\subsubsection{Parameters}
\label{\detokenize{pycatch/utils/calibration:parameters}}\begin{description}
\sphinxlineitem{map}{[}sunpy.map.Map{]}
\sphinxAtStartPar
The input solar map to which the limb correction will be applied.

\end{description}


\subsubsection{Returns}
\label{\detokenize{pycatch/utils/calibration:returns}}\begin{description}
\sphinxlineitem{sunpy.map.Map}
\sphinxAtStartPar
A solar map with annulus limb correction applied.

\end{description}

\end{fulllineitems}

\index{calibrate\_aia() (in module pycatch.utils.calibration)@\spxentry{calibrate\_aia()}\spxextra{in module pycatch.utils.calibration}}

\begin{fulllineitems}
\phantomsection\label{\detokenize{pycatch/utils/calibration:pycatch.utils.calibration.calibrate_aia}}
\pysigstartsignatures
\pysiglinewithargsret{\sphinxcode{\sphinxupquote{pycatch.utils.calibration.}}\sphinxbfcode{\sphinxupquote{calibrate\_aia}}}{\emph{\DUrole{n}{map}}, \emph{\DUrole{n}{register}\DUrole{o}{=}\DUrole{default_value}{True}}, \emph{\DUrole{n}{normalize}\DUrole{o}{=}\DUrole{default_value}{True}}, \emph{\DUrole{n}{deconvolve}\DUrole{o}{=}\DUrole{default_value}{None}}, \emph{\DUrole{n}{alc}\DUrole{o}{=}\DUrole{default_value}{True}}, \emph{\DUrole{n}{degradation}\DUrole{o}{=}\DUrole{default_value}{True}}, \emph{\DUrole{n}{cut\_limb}\DUrole{o}{=}\DUrole{default_value}{True}}}{}
\pysigstopsignatures
\sphinxAtStartPar
Calibrate and preprocess an AIA (Atmospheric Imaging Assembly) map.

\sphinxAtStartPar
This function performs various calibration and preprocessing steps on an AIA map to prepare it for further analysis.


\subsubsection{Parameters}
\label{\detokenize{pycatch/utils/calibration:id1}}\begin{description}
\sphinxlineitem{map}{[}sunpy.map.Map{]}
\sphinxAtStartPar
The input AIA map to be calibrated and preprocessed.

\sphinxlineitem{register}{[}bool, optional{]}
\sphinxAtStartPar
Whether to perform image registration. Default is True.

\sphinxlineitem{normalize}{[}bool, optional{]}
\sphinxAtStartPar
Whether to normalize the exposure. Default is True.

\sphinxlineitem{deconvolve}{[}bool or None or numpy.ndarray, optional{]}
\sphinxAtStartPar
Whether to perform PSF deconvolution. If set to True, deconvolution with aiapy.psf.deconvolve is applied.
If custom PSF array is given, it uses this array instead. If set to False, it is not applied.

\sphinxlineitem{alc}{[}bool, optional{]}
\sphinxAtStartPar
Whether to perform annulus limb correction. Default is True.

\sphinxlineitem{degradation}{[}bool, optional{]}
\sphinxAtStartPar
Whether to correct for instrument degradation. Default is True.

\sphinxlineitem{cut\_limb}{[}bool, optional{]}
\sphinxAtStartPar
Whether to cut the limb of the solar disk. Default is True.

\end{description}


\subsubsection{Returns}
\label{\detokenize{pycatch/utils/calibration:id2}}\begin{description}
\sphinxlineitem{sunpy.map.Map}
\sphinxAtStartPar
A calibrated and preprocessed AIA map ready for analysis.

\end{description}

\end{fulllineitems}

\index{calibrate\_hmi() (in module pycatch.utils.calibration)@\spxentry{calibrate\_hmi()}\spxextra{in module pycatch.utils.calibration}}

\begin{fulllineitems}
\phantomsection\label{\detokenize{pycatch/utils/calibration:pycatch.utils.calibration.calibrate_hmi}}
\pysigstartsignatures
\pysiglinewithargsret{\sphinxcode{\sphinxupquote{pycatch.utils.calibration.}}\sphinxbfcode{\sphinxupquote{calibrate\_hmi}}}{\emph{\DUrole{n}{map}}, \emph{\DUrole{n}{intensity\_map}}, \emph{\DUrole{n}{rotate}\DUrole{o}{=}\DUrole{default_value}{True}}, \emph{\DUrole{n}{align}\DUrole{o}{=}\DUrole{default_value}{True}}, \emph{\DUrole{n}{cut\_limb}\DUrole{o}{=}\DUrole{default_value}{True}}}{}
\pysigstopsignatures
\sphinxAtStartPar
Calibrate and preprocess an HMI (Helioseismic and Magnetic Imager) map.

\sphinxAtStartPar
This function performs various calibration and preprocessing steps on an HMI map to prepare it for further analysis.


\subsubsection{Parameters}
\label{\detokenize{pycatch/utils/calibration:id3}}\begin{description}
\sphinxlineitem{map}{[}sunpy.map.Map{]}
\sphinxAtStartPar
The input HMI map to be calibrated and preprocessed.

\sphinxlineitem{intensity\_map}{[}sunpy.map.Map{]}
\sphinxAtStartPar
An intensity map to align the HMI map with.

\sphinxlineitem{rotate}{[}bool, optional{]}
\sphinxAtStartPar
Whether to rotate the HMI map to have north up. Default is True.

\sphinxlineitem{align}{[}bool, optional{]}
\sphinxAtStartPar
Whether to align the HMI map with an intensity map. Default is True.

\sphinxlineitem{cut\_limb}{[}bool, optional{]}
\sphinxAtStartPar
Set off\sphinxhyphen{}limb pixel values to NaN. Default is True.

\end{description}


\subsubsection{Returns}
\label{\detokenize{pycatch/utils/calibration:id4}}\begin{description}
\sphinxlineitem{sunpy.map.Map}
\sphinxAtStartPar
A calibrated and preprocessed HMI map ready for analysis.

\end{description}

\end{fulllineitems}

\index{calibrate\_stereo() (in module pycatch.utils.calibration)@\spxentry{calibrate\_stereo()}\spxextra{in module pycatch.utils.calibration}}

\begin{fulllineitems}
\phantomsection\label{\detokenize{pycatch/utils/calibration:pycatch.utils.calibration.calibrate_stereo}}
\pysigstartsignatures
\pysiglinewithargsret{\sphinxcode{\sphinxupquote{pycatch.utils.calibration.}}\sphinxbfcode{\sphinxupquote{calibrate\_stereo}}}{\emph{\DUrole{n}{map}}, \emph{\DUrole{n}{register}\DUrole{o}{=}\DUrole{default_value}{True}}, \emph{\DUrole{n}{normalize}\DUrole{o}{=}\DUrole{default_value}{True}}, \emph{\DUrole{n}{deconvolve}\DUrole{o}{=}\DUrole{default_value}{None}}, \emph{\DUrole{n}{alc}\DUrole{o}{=}\DUrole{default_value}{True}}, \emph{\DUrole{n}{cut\_limb}\DUrole{o}{=}\DUrole{default_value}{True}}}{}
\pysigstopsignatures
\sphinxAtStartPar
Calibrate and preprocess a STEREO EUV (Solar TErrestrial RElations Observatory) map.

\sphinxAtStartPar
This function performs various calibration and preprocessing steps on a STEREO map to prepare it for further analysis.


\subsubsection{Parameters}
\label{\detokenize{pycatch/utils/calibration:id5}}\begin{description}
\sphinxlineitem{map}{[}sunpy.map.Map{]}
\sphinxAtStartPar
The input STEREO map to be calibrated and preprocessed.

\sphinxlineitem{register}{[}bool, optional{]}
\sphinxAtStartPar
Whether to perform map rotation to register the map. Default is True.

\sphinxlineitem{normalize}{[}bool, optional{]}
\sphinxAtStartPar
Whether to normalize the exposure. Default is True.

\sphinxlineitem{deconvolve}{[}bool or None or numpy.ndarray, optional{]}
\sphinxAtStartPar
Whether to perform PSF deconvolution. If set to True, deconvolution with aiapy.psf.deconvolve is applied.
If custom PSF array is given, it uses this array instead. If set to False, it is not applied.

\sphinxlineitem{alc}{[}bool, optional{]}
\sphinxAtStartPar
Whether to perform annulus limb correction. Default is True.

\sphinxlineitem{cut\_limb}{[}bool, optional{]}
\sphinxAtStartPar
Set off\sphinxhyphen{}limb pixel values to NaN. Default is True.

\end{description}


\subsubsection{Returns}
\label{\detokenize{pycatch/utils/calibration:id6}}\begin{description}
\sphinxlineitem{sunpy.map.Map}
\sphinxAtStartPar
A calibrated and preprocessed STEREO map ready for analysis.

\end{description}

\end{fulllineitems}


\sphinxstepscope


\subsection{pycatch.utils.ch\_mapping Module}
\label{\detokenize{pycatch/utils/ch_mapping:pycatch-utils-ch-mapping-module}}\label{\detokenize{pycatch/utils/ch_mapping::doc}}
\sphinxAtStartPar
The \sphinxtitleref{ch\_mapping} module provides map operation functions.

\phantomsection\label{\detokenize{pycatch/utils/ch_mapping:module-pycatch.utils.ch_mapping}}\index{module@\spxentry{module}!pycatch.utils.ch\_mapping@\spxentry{pycatch.utils.ch\_mapping}}\index{pycatch.utils.ch\_mapping@\spxentry{pycatch.utils.ch\_mapping}!module@\spxentry{module}}\index{calc\_area\_curves() (in module pycatch.utils.ch\_mapping)@\spxentry{calc\_area\_curves()}\spxextra{in module pycatch.utils.ch\_mapping}}

\begin{fulllineitems}
\phantomsection\label{\detokenize{pycatch/utils/ch_mapping:pycatch.utils.ch_mapping.calc_area_curves}}
\pysigstartsignatures
\pysiglinewithargsret{\sphinxcode{\sphinxupquote{pycatch.utils.ch\_mapping.}}\sphinxbfcode{\sphinxupquote{calc\_area\_curves}}}{\emph{\DUrole{n}{map}}, \emph{\DUrole{n}{th}}, \emph{\DUrole{n}{kernel}}, \emph{\DUrole{n}{seed}}, \emph{\DUrole{n}{minval}}, \emph{\DUrole{n}{coreg}}}{}
\pysigstopsignatures
\sphinxAtStartPar
Wrapper to calculate area curves for a coronal hole in a solar map.


\subsubsection{Parameters}
\label{\detokenize{pycatch/utils/ch_mapping:parameters}}\begin{description}
\sphinxlineitem{map}{[}sunpy.map.Map{]}
\sphinxAtStartPar
The input solar map containing coronal hole data.

\sphinxlineitem{th}{[}float{]}
\sphinxAtStartPar
The threshold value for coronal hole extraction.

\sphinxlineitem{kernel}{[}int or None{]}
\sphinxAtStartPar
The size of the circular kernel for morphological operations.

\sphinxlineitem{seed}{[}tuple{]}
\sphinxAtStartPar
The seed point coordinates (lon, lat) for coronal hole extraction.

\sphinxlineitem{minval}{[}float{]}
\sphinxAtStartPar
The minimum threshold value to consider for area calculation.

\sphinxlineitem{coreg}{[}sunpy.map.Map{]}
\sphinxAtStartPar
The coregistered solar map for area calculation.

\end{description}


\subsubsection{Returns}
\label{\detokenize{pycatch/utils/ch_mapping:returns}}\begin{description}
\sphinxlineitem{numpy.ndarray}
\sphinxAtStartPar
The area curves for the coronal hole as function of intensity.

\end{description}

\end{fulllineitems}

\index{catch\_calc() (in module pycatch.utils.ch\_mapping)@\spxentry{catch\_calc()}\spxextra{in module pycatch.utils.ch\_mapping}}

\begin{fulllineitems}
\phantomsection\label{\detokenize{pycatch/utils/ch_mapping:pycatch.utils.ch_mapping.catch_calc}}
\pysigstartsignatures
\pysiglinewithargsret{\sphinxcode{\sphinxupquote{pycatch.utils.ch\_mapping.}}\sphinxbfcode{\sphinxupquote{catch\_calc}}}{\emph{\DUrole{n}{binmaps}}, \emph{\DUrole{n}{binary}\DUrole{o}{=}\DUrole{default_value}{True}}}{}
\pysigstopsignatures
\sphinxAtStartPar
Calculate coronal hole properties from a list of binary maps.

\sphinxAtStartPar
This function calculates various properties of coronal holes, such as area, center of mass, extent, and their uncertainties, from a list of binary maps representing different levels of coronal hole segmentation.


\subsubsection{Parameters}
\label{\detokenize{pycatch/utils/ch_mapping:id1}}\begin{description}
\sphinxlineitem{binmaps}{[}list of sunpy.map.Map{]}
\sphinxAtStartPar
A list of binary maps representing different levels of coronal hole segmentation.

\sphinxlineitem{binary}{[}bool, optional{]}
\sphinxAtStartPar
Indicates whether the input binary maps are binary (Default is True).

\end{description}


\subsubsection{Returns}
\label{\detokenize{pycatch/utils/ch_mapping:id2}}\begin{description}
\sphinxlineitem{sunpy.map.Map}
\sphinxAtStartPar
A binary map representing the combined coronal hole regions.

\sphinxlineitem{float}
\sphinxAtStartPar
The mean area of coronal holes.

\sphinxlineitem{float}
\sphinxAtStartPar
The uncertainty of the mean area of coronal holes.

\sphinxlineitem{numpy.ndarray}
\sphinxAtStartPar
An array containing the mean center of mass (lon, lat) of coronal holes.

\sphinxlineitem{numpy.ndarray}
\sphinxAtStartPar
An array containing the uncertainty of the mean center of mass (lon, lat) of coronal holes.

\sphinxlineitem{numpy.ndarray}
\sphinxAtStartPar
An array containing the mean extent (lon1, lon2, lat1, lat2) of coronal holes.

\sphinxlineitem{numpy.ndarray}
\sphinxAtStartPar
An array containing the uncertainty of the mean extent (lon1, lon2, lat1, lat2) of coronal holes.

\end{description}

\end{fulllineitems}

\index{catch\_mag() (in module pycatch.utils.ch\_mapping)@\spxentry{catch\_mag()}\spxextra{in module pycatch.utils.ch\_mapping}}

\begin{fulllineitems}
\phantomsection\label{\detokenize{pycatch/utils/ch_mapping:pycatch.utils.ch_mapping.catch_mag}}
\pysigstartsignatures
\pysiglinewithargsret{\sphinxcode{\sphinxupquote{pycatch.utils.ch\_mapping.}}\sphinxbfcode{\sphinxupquote{catch\_mag}}}{\emph{\DUrole{n}{binmaps}}, \emph{\DUrole{n}{magmap}}}{}
\pysigstopsignatures
\sphinxAtStartPar
Wrapper to calculate magnetic properties of coronal holes from a list of binary maps and a magnetic field map.

\sphinxAtStartPar
This function calculates various magnetic properties of coronal holes, including signed and unsigned mean magnetic flux density, signed and unsigned magnetic flux, and flux balance, from a list of binary maps representing different levels of coronal hole segmentation and a magnetic field map.


\subsubsection{Parameters}
\label{\detokenize{pycatch/utils/ch_mapping:id3}}\begin{description}
\sphinxlineitem{binmaps}{[}list of sunpy.map.Map{]}
\sphinxAtStartPar
A list of binary maps representing different levels of coronal hole segmentation.

\sphinxlineitem{magmap}{[}sunpy.map.Map{]}
\sphinxAtStartPar
The magnetic field map.

\end{description}


\subsubsection{Returns}
\label{\detokenize{pycatch/utils/ch_mapping:id4}}\begin{description}
\sphinxlineitem{float}
\sphinxAtStartPar
The mean signed magnetic flux density of coronal holes.

\sphinxlineitem{float}
\sphinxAtStartPar
The uncertainty of the mean signed magnetic flux density of coronal holes.

\sphinxlineitem{float}
\sphinxAtStartPar
The mean unsigned magnetic flux density of coronal holes.

\sphinxlineitem{float}
\sphinxAtStartPar
The uncertainty of the mean unsigned magnetic flux density of coronal holes.

\sphinxlineitem{float}
\sphinxAtStartPar
The mean signed magnetic flux of coronal holes.

\sphinxlineitem{float}
\sphinxAtStartPar
The uncertainty of the mean signed magnetic flux of coronal holes.

\sphinxlineitem{float}
\sphinxAtStartPar
The mean unsigned magnetic flux of coronal holes.

\sphinxlineitem{float}
\sphinxAtStartPar
The uncertainty of the mean unsigned magnetic flux of coronal holes.

\sphinxlineitem{float}
\sphinxAtStartPar
The mean flux balance of coronal holes.

\sphinxlineitem{float}
\sphinxAtStartPar
The uncertainty of the mean flux balance of coronal holes.

\end{description}

\end{fulllineitems}

\index{catch\_uncertainty() (in module pycatch.utils.ch\_mapping)@\spxentry{catch\_uncertainty()}\spxextra{in module pycatch.utils.ch\_mapping}}

\begin{fulllineitems}
\phantomsection\label{\detokenize{pycatch/utils/ch_mapping:pycatch.utils.ch_mapping.catch_uncertainty}}
\pysigstartsignatures
\pysiglinewithargsret{\sphinxcode{\sphinxupquote{pycatch.utils.ch\_mapping.}}\sphinxbfcode{\sphinxupquote{catch\_uncertainty}}}{\emph{\DUrole{n}{x}}}{}
\pysigstopsignatures
\sphinxAtStartPar
Calculate the uncertainty of a given data array.

\sphinxAtStartPar
This function computes the uncertainty of a given data array using the maximum absolute deviation from the mean.


\subsubsection{Parameters}
\label{\detokenize{pycatch/utils/ch_mapping:id5}}\begin{description}
\sphinxlineitem{x}{[}numpy.ndarray{]}
\sphinxAtStartPar
The input data array for which uncertainty is calculated.

\end{description}


\subsubsection{Returns}
\label{\detokenize{pycatch/utils/ch_mapping:id6}}\begin{description}
\sphinxlineitem{float}
\sphinxAtStartPar
The uncertainty value calculated as the maximum absolute deviation from the mean.

\end{description}

\end{fulllineitems}

\index{ch\_area() (in module pycatch.utils.ch\_mapping)@\spxentry{ch\_area()}\spxextra{in module pycatch.utils.ch\_mapping}}

\begin{fulllineitems}
\phantomsection\label{\detokenize{pycatch/utils/ch_mapping:pycatch.utils.ch_mapping.ch_area}}
\pysigstartsignatures
\pysiglinewithargsret{\sphinxcode{\sphinxupquote{pycatch.utils.ch\_mapping.}}\sphinxbfcode{\sphinxupquote{ch\_area}}}{\emph{\DUrole{n}{map}}, \emph{\DUrole{n}{coreg}\DUrole{o}{=}\DUrole{default_value}{None}}, \emph{\DUrole{n}{binary}\DUrole{o}{=}\DUrole{default_value}{False}}}{}
\pysigstopsignatures
\sphinxAtStartPar
Calculate the area of a coronal hole in a solar map.


\subsubsection{Parameters}
\label{\detokenize{pycatch/utils/ch_mapping:id7}}\begin{description}
\sphinxlineitem{map}{[}sunpy.map.Map{]}
\sphinxAtStartPar
The input solar map containing coronal hole data.

\sphinxlineitem{coreg}{[}numpy.ndarray, optional{]}
\sphinxAtStartPar
An optional curvature correction factor for the map. If not provided, it will be computed internally.

\sphinxlineitem{binary}{[}bool, optional{]}
\sphinxAtStartPar
If True, the map is treated as binary data, where coronal hole pixels have a value of 1. If False, the map is thresholded to binary data. Default is False.

\end{description}


\subsubsection{Returns}
\label{\detokenize{pycatch/utils/ch_mapping:id8}}\begin{description}
\sphinxlineitem{float}
\sphinxAtStartPar
The calculated area of the coronal hole in 10\textasciicircum{}10 km\textasciicircum{}2.

\end{description}

\end{fulllineitems}

\index{ch\_flux() (in module pycatch.utils.ch\_mapping)@\spxentry{ch\_flux()}\spxextra{in module pycatch.utils.ch\_mapping}}

\begin{fulllineitems}
\phantomsection\label{\detokenize{pycatch/utils/ch_mapping:pycatch.utils.ch_mapping.ch_flux}}
\pysigstartsignatures
\pysiglinewithargsret{\sphinxcode{\sphinxupquote{pycatch.utils.ch\_mapping.}}\sphinxbfcode{\sphinxupquote{ch\_flux}}}{\emph{\DUrole{n}{binmap}}, \emph{\DUrole{n}{magmap}}, \emph{\DUrole{n}{coreg}\DUrole{o}{=}\DUrole{default_value}{{[}0{]}}}}{}
\pysigstopsignatures
\sphinxAtStartPar
Calculate magnetic properties of a coronal hole region from a binary map and a magnetic field map.

\sphinxAtStartPar
This function calculates various magnetic properties of a coronal hole region, including signed mean magnetic flux density, unsigned mean magnetic flux density, signed magnetic flux, and unsigned magnetic flux, based on a binary map representing the coronal hole region and a magnetic field map.


\subsubsection{Parameters}
\label{\detokenize{pycatch/utils/ch_mapping:id9}}\begin{description}
\sphinxlineitem{binmap}{[}sunpy.map.Map{]}
\sphinxAtStartPar
A binary map representing the coronal hole region.

\sphinxlineitem{magmap}{[}sunpy.map.Map{]}
\sphinxAtStartPar
The magnetic field map.

\sphinxlineitem{coreg}{[}list of float, optional{]}
\sphinxAtStartPar
An optional curvature correction factor for the map. If not provided, it will be computed internally.

\end{description}


\subsubsection{Returns}
\label{\detokenize{pycatch/utils/ch_mapping:id10}}\begin{description}
\sphinxlineitem{float}
\sphinxAtStartPar
The mean signed magnetic flux density of the coronal hole region.

\sphinxlineitem{float}
\sphinxAtStartPar
The mean unsigned magnetic flux density of the coronal hole region.

\sphinxlineitem{float}
\sphinxAtStartPar
The signed magnetic flux of the coronal hole region.

\sphinxlineitem{float}
\sphinxAtStartPar
The unsigned magnetic flux of the coronal hole region.

\end{description}

\end{fulllineitems}

\index{curve\_corr() (in module pycatch.utils.ch\_mapping)@\spxentry{curve\_corr()}\spxextra{in module pycatch.utils.ch\_mapping}}

\begin{fulllineitems}
\phantomsection\label{\detokenize{pycatch/utils/ch_mapping:pycatch.utils.ch_mapping.curve_corr}}
\pysigstartsignatures
\pysiglinewithargsret{\sphinxcode{\sphinxupquote{pycatch.utils.ch\_mapping.}}\sphinxbfcode{\sphinxupquote{curve\_corr}}}{\emph{\DUrole{n}{map}}}{}
\pysigstopsignatures
\sphinxAtStartPar
Compute a correction factor for curvature in a solar map.


\subsubsection{Parameters}
\label{\detokenize{pycatch/utils/ch_mapping:id11}}\begin{description}
\sphinxlineitem{map}{[}sunpy.map.Map{]}
\sphinxAtStartPar
The input solar map for which the correction factor is computed.

\end{description}


\subsubsection{Returns}
\label{\detokenize{pycatch/utils/ch_mapping:id12}}\begin{description}
\sphinxlineitem{numpy.ndarray}
\sphinxAtStartPar
A 2D array representing the correction factor for curvature in the solar map.

\end{description}

\end{fulllineitems}

\index{cutout() (in module pycatch.utils.ch\_mapping)@\spxentry{cutout()}\spxextra{in module pycatch.utils.ch\_mapping}}

\begin{fulllineitems}
\phantomsection\label{\detokenize{pycatch/utils/ch_mapping:pycatch.utils.ch_mapping.cutout}}
\pysigstartsignatures
\pysiglinewithargsret{\sphinxcode{\sphinxupquote{pycatch.utils.ch\_mapping.}}\sphinxbfcode{\sphinxupquote{cutout}}}{\emph{\DUrole{n}{map}}, \emph{\DUrole{n}{top}}, \emph{\DUrole{n}{bot}}}{}
\pysigstopsignatures
\sphinxAtStartPar
Create a cutout region from a solar map based on top and bottom coordinates.

\sphinxAtStartPar
This function generates a cutout region from the input solar map (\sphinxtitleref{map}) using specified top and bottom coordinates.


\subsubsection{Parameters}
\label{\detokenize{pycatch/utils/ch_mapping:id13}}\begin{description}
\sphinxlineitem{map}{[}sunpy.map.Map{]}
\sphinxAtStartPar
The input solar map from which the cutout region will be created.

\sphinxlineitem{top}{[}tuple{]}
\sphinxAtStartPar
A tuple containing the top\sphinxhyphen{}right corner coordinates (x, y) of the cutout region in arcseconds.

\sphinxlineitem{bot}{[}tuple{]}
\sphinxAtStartPar
A tuple containing the bottom\sphinxhyphen{}left corner coordinates (x, y) of the cutout region in arcseconds.

\end{description}


\subsubsection{Returns}
\label{\detokenize{pycatch/utils/ch_mapping:id14}}\begin{description}
\sphinxlineitem{sunpy.map.Map}
\sphinxAtStartPar
A cutout region of the input solar map based on the specified coordinates.

\end{description}

\end{fulllineitems}

\index{extract\_ch() (in module pycatch.utils.ch\_mapping)@\spxentry{extract\_ch()}\spxextra{in module pycatch.utils.ch\_mapping}}

\begin{fulllineitems}
\phantomsection\label{\detokenize{pycatch/utils/ch_mapping:pycatch.utils.ch_mapping.extract_ch}}
\pysigstartsignatures
\pysiglinewithargsret{\sphinxcode{\sphinxupquote{pycatch.utils.ch\_mapping.}}\sphinxbfcode{\sphinxupquote{extract\_ch}}}{\emph{\DUrole{n}{map}}, \emph{\DUrole{n}{thr}}, \emph{\DUrole{n}{kernel}}, \emph{\DUrole{n}{seed}}}{}
\pysigstopsignatures
\sphinxAtStartPar
Extract a coronal hole from a solar map based on a threshold and seed point.


\subsubsection{Parameters}
\label{\detokenize{pycatch/utils/ch_mapping:id15}}\begin{description}
\sphinxlineitem{map}{[}sunpy.map.Map{]}
\sphinxAtStartPar
The input solar map containing coronal hole data.

\sphinxlineitem{thr}{[}float{]}
\sphinxAtStartPar
The threshold value for coronal hole extraction.

\sphinxlineitem{kernel}{[}int or None{]}
\sphinxAtStartPar
The size of the circular kernel for morphological operations. If None, the kernel size is automatically determined based on map resolution.

\sphinxlineitem{seed}{[}tuple{]}
\sphinxAtStartPar
The seed point coordinates (x, y) for coronal hole extraction.

\end{description}


\subsubsection{Returns}
\label{\detokenize{pycatch/utils/ch_mapping:id16}}\begin{description}
\sphinxlineitem{sunpy.map.Map}
\sphinxAtStartPar
A “binary” map containing the extracted coronal hole region.

\end{description}

\end{fulllineitems}

\index{from\_5binmap() (in module pycatch.utils.ch\_mapping)@\spxentry{from\_5binmap()}\spxextra{in module pycatch.utils.ch\_mapping}}

\begin{fulllineitems}
\phantomsection\label{\detokenize{pycatch/utils/ch_mapping:pycatch.utils.ch_mapping.from_5binmap}}
\pysigstartsignatures
\pysiglinewithargsret{\sphinxcode{\sphinxupquote{pycatch.utils.ch\_mapping.}}\sphinxbfcode{\sphinxupquote{from\_5binmap}}}{\emph{\DUrole{n}{binmap}}}{}
\pysigstopsignatures
\sphinxAtStartPar
Split a 5\sphinxhyphen{}level binary map into multiple binary maps.


\subsubsection{Parameters}
\label{\detokenize{pycatch/utils/ch_mapping:id17}}\begin{description}
\sphinxlineitem{binmap}{[}sunpy.map.Map{]}
\sphinxAtStartPar
A 5\sphinxhyphen{}level binary map containing multiple threshold levels.

\end{description}


\subsubsection{Returns}
\label{\detokenize{pycatch/utils/ch_mapping:id18}}\begin{description}
\sphinxlineitem{list of sunpy.map.Map}
\sphinxAtStartPar
A list of binary maps, one for each threshold level extracted from the input 5\sphinxhyphen{}level binary map.

\end{description}

\end{fulllineitems}

\index{get\_curves() (in module pycatch.utils.ch\_mapping)@\spxentry{get\_curves()}\spxextra{in module pycatch.utils.ch\_mapping}}

\begin{fulllineitems}
\phantomsection\label{\detokenize{pycatch/utils/ch_mapping:pycatch.utils.ch_mapping.get_curves}}
\pysigstartsignatures
\pysiglinewithargsret{\sphinxcode{\sphinxupquote{pycatch.utils.ch\_mapping.}}\sphinxbfcode{\sphinxupquote{get\_curves}}}{\emph{\DUrole{n}{map}}, \emph{\DUrole{n}{seed}}, \emph{\DUrole{n}{kernel}\DUrole{o}{=}\DUrole{default_value}{None}}, \emph{\DUrole{n}{upper\_lim}\DUrole{o}{=}\DUrole{default_value}{False}}, \emph{\DUrole{n}{cores}\DUrole{o}{=}\DUrole{default_value}{8}}}{}
\pysigstopsignatures
\sphinxAtStartPar
Top\sphinxhyphen{}level wrapper to calculate coronal hole area curves for a range of threshold values.

\sphinxAtStartPar
This function calculates coronal hole area curves for a range of threshold values, optionally considering an upper limit. It also computes uncertainty in the area curves. The function uses parallel processing to speed up computation.


\subsubsection{Parameters}
\label{\detokenize{pycatch/utils/ch_mapping:id19}}\begin{description}
\sphinxlineitem{map}{[}sunpy.map.Map{]}
\sphinxAtStartPar
The input solar map containing coronal hole data.

\sphinxlineitem{seed}{[}tuple{]}
\sphinxAtStartPar
The seed point coordinates (x, y) for coronal hole extraction.

\sphinxlineitem{kernel}{[}int or None, optional{]}
\sphinxAtStartPar
The size of the circular kernel for morphological operations. Default is None.

\sphinxlineitem{upper\_lim}{[}int or False, optional{]}
\sphinxAtStartPar
The upper limit for the threshold range. If False, the limit is calculated based on the median value of the solar map data. Default is False.

\sphinxlineitem{cores}{[}int, optional{]}
\sphinxAtStartPar
The number of CPU cores to use for parallel processing. Default is 8.

\end{description}


\subsubsection{Returns}
\label{\detokenize{pycatch/utils/ch_mapping:id20}}\begin{description}
\sphinxlineitem{tuple}
\sphinxAtStartPar
A tuple containing three arrays:
\sphinxhyphen{} The threshold range.
\sphinxhyphen{} The calculated area curves for the coronal hole.
\sphinxhyphen{} The uncertainty in the area curves.

\end{description}

\end{fulllineitems}

\index{get\_intensity() (in module pycatch.utils.ch\_mapping)@\spxentry{get\_intensity()}\spxextra{in module pycatch.utils.ch\_mapping}}

\begin{fulllineitems}
\phantomsection\label{\detokenize{pycatch/utils/ch_mapping:pycatch.utils.ch_mapping.get_intensity}}
\pysigstartsignatures
\pysiglinewithargsret{\sphinxcode{\sphinxupquote{pycatch.utils.ch\_mapping.}}\sphinxbfcode{\sphinxupquote{get\_intensity}}}{\emph{\DUrole{n}{binmaps}}, \emph{\DUrole{n}{map}}}{}
\pysigstopsignatures
\sphinxAtStartPar
Calculate intensity properties of coronal holes from a list of binary maps.

\sphinxAtStartPar
This function calculates the mean and median intensity properties of coronal holes from a list of binary maps representing different levels of coronal hole segmentation.


\subsubsection{Parameters}
\label{\detokenize{pycatch/utils/ch_mapping:id21}}\begin{description}
\sphinxlineitem{binmaps}{[}list of sunpy.map.Map{]}
\sphinxAtStartPar
A list of binary maps representing different levels of coronal hole segmentation.

\sphinxlineitem{map}{[}sunpy.map.Map{]}
\sphinxAtStartPar
The original intensity map.

\end{description}


\subsubsection{Returns}
\label{\detokenize{pycatch/utils/ch_mapping:id22}}\begin{description}
\sphinxlineitem{float}
\sphinxAtStartPar
The mean intensity of coronal holes.

\sphinxlineitem{float}
\sphinxAtStartPar
The uncertainty of the mean intensity of coronal holes.

\sphinxlineitem{float}
\sphinxAtStartPar
The median intensity of coronal holes.

\sphinxlineitem{float}
\sphinxAtStartPar
The uncertainty of the median intensity of coronal holes.

\end{description}

\end{fulllineitems}

\index{make\_circle() (in module pycatch.utils.ch\_mapping)@\spxentry{make\_circle()}\spxextra{in module pycatch.utils.ch\_mapping}}

\begin{fulllineitems}
\phantomsection\label{\detokenize{pycatch/utils/ch_mapping:pycatch.utils.ch_mapping.make_circle}}
\pysigstartsignatures
\pysiglinewithargsret{\sphinxcode{\sphinxupquote{pycatch.utils.ch\_mapping.}}\sphinxbfcode{\sphinxupquote{make\_circle}}}{\emph{\DUrole{n}{s}}}{}
\pysigstopsignatures
\sphinxAtStartPar
Generate a binary circular mask.

\sphinxAtStartPar
This function generates a binary circular mask with a specified size, where the circle is centered within the mask.


\subsubsection{Parameters}
\label{\detokenize{pycatch/utils/ch_mapping:id23}}\begin{description}
\sphinxlineitem{s}{[}int{]}
\sphinxAtStartPar
The size of the circular mask (side length).

\end{description}


\subsubsection{Returns}
\label{\detokenize{pycatch/utils/ch_mapping:id24}}\begin{description}
\sphinxlineitem{numpy.ndarray}
\sphinxAtStartPar
A binary circular mask with the specified size.

\end{description}

\end{fulllineitems}

\index{min\_picker() (in module pycatch.utils.ch\_mapping)@\spxentry{min\_picker()}\spxextra{in module pycatch.utils.ch\_mapping}}

\begin{fulllineitems}
\phantomsection\label{\detokenize{pycatch/utils/ch_mapping:pycatch.utils.ch_mapping.min_picker}}
\pysigstartsignatures
\pysiglinewithargsret{\sphinxcode{\sphinxupquote{pycatch.utils.ch\_mapping.}}\sphinxbfcode{\sphinxupquote{min\_picker}}}{\emph{\DUrole{n}{seed}}, \emph{\DUrole{n}{data}}}{}
\pysigstopsignatures
\sphinxAtStartPar
Find the minimum value in a local neighborhood around a seed point in a data array.

\sphinxAtStartPar
This function identifies the minimum value within a 5x5 local neighborhood centered around a seed point (xs, ys)
in the input data array.


\subsubsection{Parameters}
\label{\detokenize{pycatch/utils/ch_mapping:id25}}\begin{description}
\sphinxlineitem{seed}{[}tuple{]}
\sphinxAtStartPar
The seed point coordinates (xs, ys) as a tuple.

\sphinxlineitem{data}{[}array\sphinxhyphen{}like{]}
\sphinxAtStartPar
The input data array in which to search for the minimum value.

\end{description}


\subsubsection{Returns}
\label{\detokenize{pycatch/utils/ch_mapping:id26}}\begin{description}
\sphinxlineitem{float}
\sphinxAtStartPar
The minimum value within the local neighborhood.

\end{description}

\end{fulllineitems}

\index{to\_5binmap() (in module pycatch.utils.ch\_mapping)@\spxentry{to\_5binmap()}\spxextra{in module pycatch.utils.ch\_mapping}}

\begin{fulllineitems}
\phantomsection\label{\detokenize{pycatch/utils/ch_mapping:pycatch.utils.ch_mapping.to_5binmap}}
\pysigstartsignatures
\pysiglinewithargsret{\sphinxcode{\sphinxupquote{pycatch.utils.ch\_mapping.}}\sphinxbfcode{\sphinxupquote{to\_5binmap}}}{\emph{\DUrole{n}{binmaps}}}{}
\pysigstopsignatures
\sphinxAtStartPar
Combine multiple binary maps into a single 5\sphinxhyphen{}level binary map.


\subsubsection{Parameters}
\label{\detokenize{pycatch/utils/ch_mapping:id27}}\begin{description}
\sphinxlineitem{binmaps}{[}list of sunpy.map.Map{]}
\sphinxAtStartPar
A list of binary maps to be combined.

\end{description}


\subsubsection{Returns}
\label{\detokenize{pycatch/utils/ch_mapping:id28}}\begin{description}
\sphinxlineitem{sunpy.map.Map}
\sphinxAtStartPar
A single 5\sphinxhyphen{}level binary map where each level represents a different threshold value.

\end{description}

\end{fulllineitems}


\sphinxstepscope


\subsection{pycatch.utils.extensions Module}
\label{\detokenize{pycatch/utils/extensions:pycatch-utils-extensions-module}}\label{\detokenize{pycatch/utils/extensions::doc}}
\sphinxAtStartPar
The \sphinxtitleref{extensions} module provides additional functions.

\phantomsection\label{\detokenize{pycatch/utils/extensions:module-pycatch.utils.extensions}}\index{module@\spxentry{module}!pycatch.utils.extensions@\spxentry{pycatch.utils.extensions}}\index{pycatch.utils.extensions@\spxentry{pycatch.utils.extensions}!module@\spxentry{module}}\index{congrid() (in module pycatch.utils.extensions)@\spxentry{congrid()}\spxextra{in module pycatch.utils.extensions}}

\begin{fulllineitems}
\phantomsection\label{\detokenize{pycatch/utils/extensions:pycatch.utils.extensions.congrid}}
\pysigstartsignatures
\pysiglinewithargsret{\sphinxcode{\sphinxupquote{pycatch.utils.extensions.}}\sphinxbfcode{\sphinxupquote{congrid}}}{\emph{\DUrole{n}{a}}, \emph{\DUrole{n}{newdims}}, \emph{\DUrole{n}{method}\DUrole{o}{=}\DUrole{default_value}{\textquotesingle{}linear\textquotesingle{}}}, \emph{\DUrole{n}{centre}\DUrole{o}{=}\DUrole{default_value}{False}}, \emph{\DUrole{n}{minusone}\DUrole{o}{=}\DUrole{default_value}{False}}}{}
\pysigstopsignatures
\sphinxAtStartPar
Resample an array to new dimension sizes using various interpolation methods.


\subsubsection{Parameters}
\label{\detokenize{pycatch/utils/extensions:parameters}}\begin{description}
\sphinxlineitem{a}{[}array\sphinxhyphen{}like{]}
\sphinxAtStartPar
The input array to be resampled.

\sphinxlineitem{newdims}{[}tuple{]}
\sphinxAtStartPar
The new dimensions to which the array should be resampled.

\sphinxlineitem{method}{[}str, optional{]}
\sphinxAtStartPar
The interpolation method to use. Default is ‘linear’.

\sphinxlineitem{centre}{[}bool, optional{]}
\sphinxAtStartPar
Whether interpolation points are at the centers of the bins. Default is False.

\sphinxlineitem{minusone}{[}bool, optional{]}
\sphinxAtStartPar
Whether to prevent extrapolation one element beyond the bounds of the input array. Default is False.

\end{description}


\subsubsection{Returns}
\label{\detokenize{pycatch/utils/extensions:returns}}\begin{description}
\sphinxlineitem{array\sphinxhyphen{}like}
\sphinxAtStartPar
The resampled array with the specified dimensions and interpolation method applied.

\end{description}


\subsubsection{Notes}
\label{\detokenize{pycatch/utils/extensions:notes}}
\sphinxAtStartPar
This function is adapted from IDL’s \sphinxtitleref{congrid} routine.
Arbitrary resampling of source array to new dimension sizes. Currently only supports maintaining the same number of dimensions. To use 1\sphinxhyphen{}D arrays, first promote them to shape (x,1).
Uses the same parameters and creates the same co\sphinxhyphen{}ordinate lookup points as IDL’’s congrid routine, which apparently originally came from a VAX/VMS routine of the same name.

\sphinxAtStartPar
method:
neighbour \sphinxhyphen{} closest value from original data
nearest and linear \sphinxhyphen{} uses n x 1\sphinxhyphen{}D interpolations using scipy.interpolate.interp1d (see Numerical Recipes for validity of use of n 1\sphinxhyphen{}D interpolations)
spline \sphinxhyphen{} uses ndimage.map\_coordinates

\sphinxAtStartPar
centre:
True \sphinxhyphen{} interpolation points are at the centres of the bins
False \sphinxhyphen{} points are at the front edge of the bin

\sphinxAtStartPar
minusone:
For example\sphinxhyphen{} inarray.shape = (i,j) \& new dimensions = (x,y)
False \sphinxhyphen{} inarray is resampled by factors of (i/x) * (j/y)
True \sphinxhyphen{} inarray is resampled by(i\sphinxhyphen{}1)/(x\sphinxhyphen{}1) * (j\sphinxhyphen{}1)/(y\sphinxhyphen{}1)
This prevents extrapolation one element beyond bounds of input array.

\end{fulllineitems}

\index{find\_nearest() (in module pycatch.utils.extensions)@\spxentry{find\_nearest()}\spxextra{in module pycatch.utils.extensions}}

\begin{fulllineitems}
\phantomsection\label{\detokenize{pycatch/utils/extensions:pycatch.utils.extensions.find_nearest}}
\pysigstartsignatures
\pysiglinewithargsret{\sphinxcode{\sphinxupquote{pycatch.utils.extensions.}}\sphinxbfcode{\sphinxupquote{find\_nearest}}}{\emph{\DUrole{n}{array}}, \emph{\DUrole{n}{value}}}{}
\pysigstopsignatures
\sphinxAtStartPar
Find the index of the nearest value in an array to a specified value.


\subsubsection{Parameters}
\label{\detokenize{pycatch/utils/extensions:id1}}\begin{description}
\sphinxlineitem{array}{[}array\sphinxhyphen{}like{]}
\sphinxAtStartPar
The input array in which to find the nearest value.

\sphinxlineitem{value}{[}float{]}
\sphinxAtStartPar
The value to which the nearest element in the array is sought.

\end{description}


\subsubsection{Returns}
\label{\detokenize{pycatch/utils/extensions:id2}}\begin{description}
\sphinxlineitem{int}
\sphinxAtStartPar
The index of the nearest value in the array.

\end{description}

\end{fulllineitems}

\index{get\_extent() (in module pycatch.utils.extensions)@\spxentry{get\_extent()}\spxextra{in module pycatch.utils.extensions}}

\begin{fulllineitems}
\phantomsection\label{\detokenize{pycatch/utils/extensions:pycatch.utils.extensions.get_extent}}
\pysigstartsignatures
\pysiglinewithargsret{\sphinxcode{\sphinxupquote{pycatch.utils.extensions.}}\sphinxbfcode{\sphinxupquote{get\_extent}}}{\emph{\DUrole{n}{map}}}{}
\pysigstopsignatures
\sphinxAtStartPar
Calculate the extent of a coronal hole in Helioprojective Cartesian (HPC) coordinates.


\subsubsection{Parameters}
\label{\detokenize{pycatch/utils/extensions:id3}}\begin{description}
\sphinxlineitem{map}{[}sunpy.map.Map{]}
\sphinxAtStartPar
A SunPy map object for which the extent needs to be calculated.

\end{description}


\subsubsection{Returns}
\label{\detokenize{pycatch/utils/extensions:id4}}\begin{description}
\sphinxlineitem{tuple}
\sphinxAtStartPar
A tuple containing two tuples representing the lower\sphinxhyphen{}left and upper\sphinxhyphen{}right corners of the map’s extent in HPC coordinates.
The format of the outer tuple is ((x\_min, y\_min), (x\_max, y\_max)), where:
(x\_min, y\_min) represents the HPC coordinates of the lower\sphinxhyphen{}left corner.
(x\_max, y\_max) represents the HPC coordinates of the upper\sphinxhyphen{}right corner.

\end{description}

\end{fulllineitems}

\index{init\_props() (in module pycatch.utils.extensions)@\spxentry{init\_props()}\spxextra{in module pycatch.utils.extensions}}

\begin{fulllineitems}
\phantomsection\label{\detokenize{pycatch/utils/extensions:pycatch.utils.extensions.init_props}}
\pysigstartsignatures
\pysiglinewithargsret{\sphinxcode{\sphinxupquote{pycatch.utils.extensions.}}\sphinxbfcode{\sphinxupquote{init\_props}}}{}{}
\pysigstopsignatures
\sphinxAtStartPar
Initialize a dictionary of properties and their units for pyCATCH.


\subsubsection{Returns}
\label{\detokenize{pycatch/utils/extensions:id5}}\begin{description}
\sphinxlineitem{dict}
\sphinxAtStartPar
A dictionary where keys are property abbreviations, and values are tuples containing the full property name and its unit.

\end{description}

\end{fulllineitems}

\index{median\_disk() (in module pycatch.utils.extensions)@\spxentry{median\_disk()}\spxextra{in module pycatch.utils.extensions}}

\begin{fulllineitems}
\phantomsection\label{\detokenize{pycatch/utils/extensions:pycatch.utils.extensions.median_disk}}
\pysigstartsignatures
\pysiglinewithargsret{\sphinxcode{\sphinxupquote{pycatch.utils.extensions.}}\sphinxbfcode{\sphinxupquote{median\_disk}}}{\emph{\DUrole{n}{map}}}{}
\pysigstopsignatures
\sphinxAtStartPar
Calculate the median value within the solar disk region of a map.


\subsubsection{Parameters}
\label{\detokenize{pycatch/utils/extensions:id6}}\begin{description}
\sphinxlineitem{map}{[}sunpy.map.Map{]}
\sphinxAtStartPar
The input solar map.

\end{description}


\subsubsection{Returns}
\label{\detokenize{pycatch/utils/extensions:id7}}\begin{description}
\sphinxlineitem{float}
\sphinxAtStartPar
The median value of the data within the solar disk.

\end{description}

\end{fulllineitems}

\index{printtxt() (in module pycatch.utils.extensions)@\spxentry{printtxt()}\spxextra{in module pycatch.utils.extensions}}

\begin{fulllineitems}
\phantomsection\label{\detokenize{pycatch/utils/extensions:pycatch.utils.extensions.printtxt}}
\pysigstartsignatures
\pysiglinewithargsret{\sphinxcode{\sphinxupquote{pycatch.utils.extensions.}}\sphinxbfcode{\sphinxupquote{printtxt}}}{\emph{\DUrole{n}{file}}, \emph{\DUrole{n}{pdict}}, \emph{\DUrole{n}{names}}, \emph{\DUrole{n}{version}}}{}
\pysigstopsignatures
\sphinxAtStartPar
Write property data to a text file.

\sphinxAtStartPar
This function writes property data to a text file with a specific format. It includes the version number,
property names, and units as headers followed by the property values.


\subsubsection{Parameters}
\label{\detokenize{pycatch/utils/extensions:id8}}
\sphinxAtStartPar
file : str
The filepath to the output text file.
pdict : dict
A dictionary containing property data, where keys correspond to property abbreviations.
names : dict
A dictionary mapping property abbreviations to tuples containing full property names and units.
version : str
The version number of the pyCATCH software.


\subsubsection{Returns}
\label{\detokenize{pycatch/utils/extensions:id9}}
\sphinxAtStartPar
None

\end{fulllineitems}


\sphinxstepscope


\subsection{pycatch.utils.plot Module}
\label{\detokenize{pycatch/utils/plot:pycatch-utils-plot-module}}\label{\detokenize{pycatch/utils/plot::doc}}
\sphinxAtStartPar
The \sphinxtitleref{plot} module provides plotting functions.

\phantomsection\label{\detokenize{pycatch/utils/plot:module-pycatch.utils.plot}}\index{module@\spxentry{module}!pycatch.utils.plot@\spxentry{pycatch.utils.plot}}\index{pycatch.utils.plot@\spxentry{pycatch.utils.plot}!module@\spxentry{module}}\index{SnappingCursor (class in pycatch.utils.plot)@\spxentry{SnappingCursor}\spxextra{class in pycatch.utils.plot}}

\begin{fulllineitems}
\phantomsection\label{\detokenize{pycatch/utils/plot:pycatch.utils.plot.SnappingCursor}}
\pysigstartsignatures
\pysiglinewithargsret{\sphinxbfcode{\sphinxupquote{class\DUrole{w}{  }}}\sphinxcode{\sphinxupquote{pycatch.utils.plot.}}\sphinxbfcode{\sphinxupquote{SnappingCursor}}}{\emph{\DUrole{n}{fig}}, \emph{\DUrole{n}{ax}}, \emph{\DUrole{n}{line}}, \emph{\DUrole{n}{line2}\DUrole{o}{=}\DUrole{default_value}{None}}, \emph{\DUrole{n}{names}\DUrole{o}{=}\DUrole{default_value}{{[}\textquotesingle{}y\textquotesingle{}{]}}}, \emph{\DUrole{n}{xe}\DUrole{o}{=}\DUrole{default_value}{None}}, \emph{\DUrole{n}{ye}\DUrole{o}{=}\DUrole{default_value}{None}}}{}
\pysigstopsignatures
\sphinxAtStartPar
Bases: \sphinxcode{\sphinxupquote{object}}

\sphinxAtStartPar
A cross\sphinxhyphen{}hair cursor that snaps to the data point of a line, which is
closest to the x position of the cursor.

\sphinxAtStartPar
For simplicity, this assumes that x values of the data are sorted.


\subsubsection{Parameters}
\label{\detokenize{pycatch/utils/plot:parameters}}\begin{description}
\sphinxlineitem{fig}{[}matplotlib.figure.Figure{]}
\sphinxAtStartPar
The matplotlib figure to which the cursor is attached.

\sphinxlineitem{ax}{[}matplotlib.axes.Axes{]}
\sphinxAtStartPar
The matplotlib axes to which the cursor is attached.

\sphinxlineitem{line}{[}matplotlib.lines.Line2D{]}
\sphinxAtStartPar
The line for which the cursor will snap to data points.

\sphinxlineitem{line2}{[}matplotlib.lines.Line2D, optional{]}
\sphinxAtStartPar
An optional second line for which the cursor can snap to data points.

\sphinxlineitem{names}{[}list of str, optional{]}
\sphinxAtStartPar
A list of names for the y\sphinxhyphen{}values displayed in the cursor’s tooltip.

\sphinxlineitem{xe}{[}array\sphinxhyphen{}like, optional{]}
\sphinxAtStartPar
An array of x\sphinxhyphen{}values associated with the data points.

\sphinxlineitem{ye}{[}array\sphinxhyphen{}like, optional{]}
\sphinxAtStartPar
An array of y\sphinxhyphen{}values associated with the data points.

\end{description}


\subsubsection{Attributes}
\label{\detokenize{pycatch/utils/plot:attributes}}\begin{description}
\sphinxlineitem{location}{[}float{]}
\sphinxAtStartPar
The x\sphinxhyphen{}value of the currently snapped data point.

\end{description}


\subsubsection{Methods}
\label{\detokenize{pycatch/utils/plot:methods}}\begin{description}
\sphinxlineitem{on\_mouse\_click(event)}
\sphinxAtStartPar
Handles mouse clicks to capture the cursor’s location.

\sphinxlineitem{on\_mouse\_move(event)}
\sphinxAtStartPar
Handles mouse movement to snap the cursor to the nearest data point.

\sphinxlineitem{on\_draw(event)}
\sphinxAtStartPar
Handles drawing events to update the cursor’s background.

\end{description}

\end{fulllineitems}

\index{get\_point\_from\_map() (in module pycatch.utils.plot)@\spxentry{get\_point\_from\_map()}\spxextra{in module pycatch.utils.plot}}

\begin{fulllineitems}
\phantomsection\label{\detokenize{pycatch/utils/plot:pycatch.utils.plot.get_point_from_map}}
\pysigstartsignatures
\pysiglinewithargsret{\sphinxcode{\sphinxupquote{pycatch.utils.plot.}}\sphinxbfcode{\sphinxupquote{get\_point\_from\_map}}}{\emph{\DUrole{n}{map}}, \emph{\DUrole{n}{hint}}, \emph{\DUrole{n}{fsize}}}{}
\pysigstopsignatures
\sphinxAtStartPar
Display a solar map and allow the user to interactively select a point on the map.


\subsubsection{Parameters}
\label{\detokenize{pycatch/utils/plot:id1}}\begin{description}
\sphinxlineitem{map}{[}sunpy.map.GenericMap{]}
\sphinxAtStartPar
The solar map to be displayed.

\sphinxlineitem{fsize}{[}tuple of int{]}
\sphinxAtStartPar
The size of the figure (width, height) in inches.

\end{description}


\subsubsection{Returns}
\label{\detokenize{pycatch/utils/plot:returns}}\begin{description}
\sphinxlineitem{point}{[}list of float{]}
\sphinxAtStartPar
A list containing the coordinates (x, y) of the selected point on the solar map.

\end{description}

\end{fulllineitems}

\index{get\_thr\_from\_curves() (in module pycatch.utils.plot)@\spxentry{get\_thr\_from\_curves()}\spxextra{in module pycatch.utils.plot}}

\begin{fulllineitems}
\phantomsection\label{\detokenize{pycatch/utils/plot:pycatch.utils.plot.get_thr_from_curves}}
\pysigstartsignatures
\pysiglinewithargsret{\sphinxcode{\sphinxupquote{pycatch.utils.plot.}}\sphinxbfcode{\sphinxupquote{get\_thr\_from\_curves}}}{\emph{\DUrole{n}{map}}, \emph{\DUrole{n}{curves}}, \emph{\DUrole{n}{fsize}}}{}
\pysigstopsignatures
\sphinxAtStartPar
Interactively get a threshold value from a plot of coronal hole area curves.


\subsubsection{Parameters}
\label{\detokenize{pycatch/utils/plot:id2}}\begin{description}
\sphinxlineitem{map}{[}sunpy.map.GenericMap{]}
\sphinxAtStartPar
The solar map for which the threshold will be determined.

\sphinxlineitem{curves}{[}tuple{]}
\sphinxAtStartPar
A tuple containing the data for plotting the curves, where \sphinxtitleref{curves{[}0{]}} is
the x\sphinxhyphen{}axis data, \sphinxtitleref{curves{[}1{]}} is the y\sphinxhyphen{}axis data for the first curve, and
\sphinxtitleref{curves{[}2{]}} is the y\sphinxhyphen{}axis data for the second curve.

\sphinxlineitem{fsize}{[}tuple of int{]}
\sphinxAtStartPar
The size of the figure (width, height) in inches.

\end{description}


\subsubsection{Returns}
\label{\detokenize{pycatch/utils/plot:id3}}\begin{description}
\sphinxlineitem{thr}{[}float{]}
\sphinxAtStartPar
The threshold value determined interactively from the plot.

\end{description}

\end{fulllineitems}

\index{get\_thr\_from\_hist() (in module pycatch.utils.plot)@\spxentry{get\_thr\_from\_hist()}\spxextra{in module pycatch.utils.plot}}

\begin{fulllineitems}
\phantomsection\label{\detokenize{pycatch/utils/plot:pycatch.utils.plot.get_thr_from_hist}}
\pysigstartsignatures
\pysiglinewithargsret{\sphinxcode{\sphinxupquote{pycatch.utils.plot.}}\sphinxbfcode{\sphinxupquote{get\_thr\_from\_hist}}}{\emph{\DUrole{n}{map}}, \emph{\DUrole{n}{fsize}}}{}
\pysigstopsignatures
\sphinxAtStartPar
Interacticely get a threshold value from a histogram of solar disk data.


\subsubsection{Parameters}
\label{\detokenize{pycatch/utils/plot:id4}}\begin{description}
\sphinxlineitem{map}{[}sunpy.map.GenericMap{]}
\sphinxAtStartPar
The solar map for which the threshold will be determined.

\sphinxlineitem{fsize}{[}tuple of int{]}
\sphinxAtStartPar
The size of the figure (width, height) in inches.

\end{description}


\subsubsection{Returns}
\label{\detokenize{pycatch/utils/plot:id5}}\begin{description}
\sphinxlineitem{thr}{[}float{]}
\sphinxAtStartPar
The threshold value determined interactively from the histogram.

\end{description}

\end{fulllineitems}

\index{plot\_map() (in module pycatch.utils.plot)@\spxentry{plot\_map()}\spxextra{in module pycatch.utils.plot}}

\begin{fulllineitems}
\phantomsection\label{\detokenize{pycatch/utils/plot:pycatch.utils.plot.plot_map}}
\pysigstartsignatures
\pysiglinewithargsret{\sphinxcode{\sphinxupquote{pycatch.utils.plot.}}\sphinxbfcode{\sphinxupquote{plot\_map}}}{\emph{\DUrole{n}{map}}, \emph{\DUrole{n}{bmap}}, \emph{\DUrole{n}{boundary}}, \emph{\DUrole{n}{uncertainty}}, \emph{\DUrole{n}{fsize}}, \emph{\DUrole{n}{save}}, \emph{\DUrole{n}{spath}}, \emph{\DUrole{n}{grid}}, \emph{\DUrole{o}{**}\DUrole{n}{kwargs}}}{}
\pysigstopsignatures
\sphinxAtStartPar
Plot a solar map with optional boundaries and uncertainty overlays.


\subsubsection{Parameters}
\label{\detokenize{pycatch/utils/plot:id6}}\begin{description}
\sphinxlineitem{map}{[}sunpy.map.GenericMap{]}
\sphinxAtStartPar
The solar map to be plotted.

\sphinxlineitem{bmap}{[}sunpy.map.GenericMap or None{]}
\sphinxAtStartPar
A binary mask map for boundaries or uncertainty, or None if not used.

\sphinxlineitem{boundary}{[}bool{]}
\sphinxAtStartPar
Whether to plot boundaries on the map.

\sphinxlineitem{uncertainty}{[}bool{]}
\sphinxAtStartPar
Whether to overlay uncertainty information on the map.

\sphinxlineitem{fsize}{[}tuple of int{]}
\sphinxAtStartPar
The size of the figure (width, height) in inches.

\sphinxlineitem{save}{[}bool{]}
\sphinxAtStartPar
Whether to save the figure to a file.

\sphinxlineitem{spath}{[}str{]}
\sphinxAtStartPar
The path to save the figure if \sphinxtitleref{save} is True.

\sphinxlineitem{{\color{red}\bfseries{}**}kwargs}
\sphinxAtStartPar
Additional keyword arguments to pass to the \sphinxtitleref{sunpy.map.Map.plot} function.

\end{description}


\subsubsection{Returns}
\label{\detokenize{pycatch/utils/plot:id9}}
\sphinxAtStartPar
None

\end{fulllineitems}


\sphinxstepscope


\chapter{Changelog}
\label{\detokenize{changelog:changelog}}\label{\detokenize{changelog::doc}}

\section{0.2.1 (Novemeber 9, 2023)}
\label{\detokenize{changelog:novemeber-9-2023}}\begin{itemize}
\item {} \begin{description}
\sphinxlineitem{Bug fixing}\begin{itemize}
\item {} 
\sphinxAtStartPar
fixed a bug where importing pycatch would not find the \_version.py file

\item {} 
\sphinxAtStartPar
fixed some path issue in the initialization of the home path

\item {} 
\sphinxAtStartPar
fixed what version was output in the .print\_properties() function

\item {} 
\sphinxAtStartPar
fixed some function descriptions

\item {} 
\sphinxAtStartPar
fixed a problem where the .print\_properties() function would not write a file if no magnetic properties were calculated.

\item {} 
\sphinxAtStartPar
fixed an issue with where possibly additional windows open when trying to select a seed point

\end{itemize}

\end{description}

\item {} \begin{description}
\sphinxlineitem{Minor changes}\begin{itemize}
\item {} 
\sphinxAtStartPar
changed the keyword order in the  .load()  routine to make it more intuitive

\item {} 
\sphinxAtStartPar
added aiapy to the list of required packages

\item {} 
\sphinxAtStartPar
switched default for .plot\_map() from small from True to False

\item {} 
\sphinxAtStartPar
added a pdf version of the documentation (User\_Manual.pdf that can be found on the github page)

\end{itemize}

\end{description}

\end{itemize}


\section{0.2.0 (September 2023)}
\label{\detokenize{changelog:september-2023}}\begin{itemize}
\item {} 
\sphinxAtStartPar
Initial functional beta release (beta \sphinxhyphen{} testing)

\end{itemize}


\section{0.1.0 (September 2023)}
\label{\detokenize{changelog:id1}}\begin{itemize}
\item {} 
\sphinxAtStartPar
Initial alpha build

\end{itemize}

\sphinxstepscope


\chapter{Acknowledging or Citing pyCATCH}
\label{\detokenize{acknowledging:acknowledging-or-citing-pycatch}}\label{\detokenize{acknowledging::doc}}
\sphinxAtStartPar
If you use pyCATCH in your scientific work, we would appreciate citing and acknowledging it in your publications.


\section{DISCLAIMER}
\label{\detokenize{acknowledging:disclaimer}}
\sphinxAtStartPar
If you use pyCATCH versions \textless{} 1.0.0 (unpublished version), please get in contact with Stephan G. Heinemann (\sphinxhref{mailto:stephan.heinemann@hmail.at}{stephan.heinemann@hmail.at}) before publication.


\subsection{Citing pyCATCH in Publications}
\label{\detokenize{acknowledging:citing-pycatch-in-publications}}
\sphinxAtStartPar
Please add the following line within your methods, conclusion or acknowledgements sections:
\begin{quote}

\sphinxAtStartPar
\sphinxstyleemphasis{This research used version X.Y.Z of the pyCATCH open source software package.}
\end{quote}

\sphinxAtStartPar
The package citation should be to the \sphinxhref{https://link.springer.com/article/10.1007/s11207-019-1539-y}{CATCH paper}, with the disclaimer that the python implementation was used.

\begin{sphinxVerbatim}[commandchars=\\\{\}]
\PYG{n+nc}{@ARTICLE}\PYG{p}{\PYGZob{}}\PYG{n+nl}{2019SoPh..294..144H}\PYG{p}{,}
\PYG{+w}{       }\PYG{n+na}{author}\PYG{+w}{ }\PYG{p}{=}\PYG{+w}{ }\PYG{l+s}{\PYGZob{}}\PYG{l+s}{\PYGZob{}}\PYG{l+s}{Heinemann}\PYG{l+s}{\PYGZcb{}}\PYG{l+s}{, Stephan G. and }\PYG{l+s}{\PYGZob{}}\PYG{l+s}{Temmer}\PYG{l+s}{\PYGZcb{}}\PYG{l+s}{, Manuela and }\PYG{l+s}{\PYGZob{}}\PYG{l+s}{Heinemann}\PYG{l+s}{\PYGZcb{}}\PYG{l+s}{, Niko and }\PYG{l+s}{\PYGZob{}}\PYG{l+s}{Dissauer}\PYG{l+s}{\PYGZcb{}}\PYG{l+s}{, Karin and }\PYG{l+s}{\PYGZob{}}\PYG{l+s}{Samara}\PYG{l+s}{\PYGZcb{}}\PYG{l+s}{, Evangelia and }\PYG{l+s}{\PYGZob{}}\PYG{l+s}{Jer}\PYG{l+s}{\PYGZob{}}\PYG{l+s}{\PYGZbs{}v}\PYG{l+s}{\PYGZob{}}\PYG{l+s}{c}\PYG{l+s}{\PYGZcb{}}\PYG{l+s}{\PYGZcb{}}\PYG{l+s}{i}\PYG{l+s}{\PYGZob{}}\PYG{l+s}{\PYGZbs{}\PYGZsq{}c}\PYG{l+s}{\PYGZcb{}}\PYG{l+s}{\PYGZcb{}}\PYG{l+s}{, Veronika and }\PYG{l+s}{\PYGZob{}}\PYG{l+s}{Hofmeister}\PYG{l+s}{\PYGZcb{}}\PYG{l+s}{, Stefan J. and }\PYG{l+s}{\PYGZob{}}\PYG{l+s}{Veronig}\PYG{l+s}{\PYGZcb{}}\PYG{l+s}{, Astrid M.}\PYG{l+s}{\PYGZcb{}}\PYG{p}{,}
\PYG{+w}{        }\PYG{n+na}{title}\PYG{+w}{ }\PYG{p}{=}\PYG{+w}{ }\PYG{l+s}{\PYGZdq{}}\PYG{l+s}{\PYGZob{}}\PYG{l+s}{Statistical Analysis and Catalog of Non\PYGZhy{}polar Coronal Holes Covering the SDO\PYGZhy{}Era Using CATCH}\PYG{l+s}{\PYGZcb{}}\PYG{l+s}{\PYGZdq{}}\PYG{p}{,}
\PYG{+w}{      }\PYG{n+na}{journal}\PYG{+w}{ }\PYG{p}{=}\PYG{+w}{ }\PYG{l+s}{\PYGZob{}}\PYG{l+s}{\PYGZbs{}solphys}\PYG{l+s}{\PYGZcb{}}\PYG{p}{,}
\PYG{+w}{     }\PYG{n+na}{keywords}\PYG{+w}{ }\PYG{p}{=}\PYG{+w}{ }\PYG{l+s}{\PYGZob{}}\PYG{l+s}{Coronal holes, Magnetic fields, Photosphere, Solar cycle, Observations, Astrophysics \PYGZhy{} Solar and Stellar Astrophysics}\PYG{l+s}{\PYGZcb{}}\PYG{p}{,}
\PYG{+w}{         }\PYG{n+na}{year}\PYG{+w}{ }\PYG{p}{=}\PYG{+w}{ }\PYG{l+m}{2019}\PYG{p}{,}
\PYG{+w}{        }\PYG{n+na}{month}\PYG{+w}{ }\PYG{p}{=}\PYG{+w}{ }\PYG{n+nv}{oct}\PYG{p}{,}
\PYG{+w}{       }\PYG{n+na}{volume}\PYG{+w}{ }\PYG{p}{=}\PYG{+w}{ }\PYG{l+s}{\PYGZob{}}\PYG{l+s}{294}\PYG{l+s}{\PYGZcb{}}\PYG{p}{,}
\PYG{+w}{       }\PYG{n+na}{number}\PYG{+w}{ }\PYG{p}{=}\PYG{+w}{ }\PYG{l+s}{\PYGZob{}}\PYG{l+s}{10}\PYG{l+s}{\PYGZcb{}}\PYG{p}{,}
\PYG{+w}{          }\PYG{n+na}{eid}\PYG{+w}{ }\PYG{p}{=}\PYG{+w}{ }\PYG{l+s}{\PYGZob{}}\PYG{l+s}{144}\PYG{l+s}{\PYGZcb{}}\PYG{p}{,}
\PYG{+w}{        }\PYG{n+na}{pages}\PYG{+w}{ }\PYG{p}{=}\PYG{+w}{ }\PYG{l+s}{\PYGZob{}}\PYG{l+s}{144}\PYG{l+s}{\PYGZcb{}}\PYG{p}{,}
\PYG{+w}{          }\PYG{n+na}{doi}\PYG{+w}{ }\PYG{p}{=}\PYG{+w}{ }\PYG{l+s}{\PYGZob{}}\PYG{l+s}{10.1007/s11207\PYGZhy{}019\PYGZhy{}1539\PYGZhy{}y}\PYG{l+s}{\PYGZcb{}}\PYG{p}{,}
\PYG{n+na}{archivePrefix}\PYG{+w}{ }\PYG{p}{=}\PYG{+w}{ }\PYG{l+s}{\PYGZob{}}\PYG{l+s}{arXiv}\PYG{l+s}{\PYGZcb{}}\PYG{p}{,}
\PYG{+w}{       }\PYG{n+na}{eprint}\PYG{+w}{ }\PYG{p}{=}\PYG{+w}{ }\PYG{l+s}{\PYGZob{}}\PYG{l+s}{1907.01990}\PYG{l+s}{\PYGZcb{}}\PYG{p}{,}
\PYG{+w}{ }\PYG{n+na}{primaryClass}\PYG{+w}{ }\PYG{p}{=}\PYG{+w}{ }\PYG{l+s}{\PYGZob{}}\PYG{l+s}{astro\PYGZhy{}ph.SR}\PYG{l+s}{\PYGZcb{}}\PYG{p}{,}
\PYG{+w}{       }\PYG{n+na}{adsurl}\PYG{+w}{ }\PYG{p}{=}\PYG{+w}{ }\PYG{l+s}{\PYGZob{}}\PYG{l+s}{https://ui.adsabs.harvard.edu/abs/2019SoPh..294..144H}\PYG{l+s}{\PYGZcb{}}\PYG{p}{,}
\PYG{+w}{      }\PYG{n+na}{adsnote}\PYG{+w}{ }\PYG{p}{=}\PYG{+w}{ }\PYG{l+s}{\PYGZob{}}\PYG{l+s}{Provided by the SAO/NASA Astrophysics Data System}\PYG{l+s}{\PYGZcb{}}
\PYG{p}{\PYGZcb{}}
\end{sphinxVerbatim}


\chapter{References}
\label{\detokenize{index:references}}
\sphinxAtStartPar
Heinemann, S.G., Temmer, M., Heinemann, N., Dissauer, K., Samara, E., Jerčić, V., Hofmeister, S.J., Veronig, A.M.: 2019, Statistical analysis and catalog of non\sphinxhyphen{}polar coronal holes covering the SDO\sphinxhyphen{}era using CATCH. Solar Phys. 294, 144.


\renewcommand{\indexname}{Python Module Index}
\begin{sphinxtheindex}
\let\bigletter\sphinxstyleindexlettergroup
\bigletter{p}
\item\relax\sphinxstyleindexentry{pycatch.utils.calibration}\sphinxstyleindexpageref{pycatch/utils/calibration:\detokenize{module-pycatch.utils.calibration}}
\item\relax\sphinxstyleindexentry{pycatch.utils.ch\_mapping}\sphinxstyleindexpageref{pycatch/utils/ch_mapping:\detokenize{module-pycatch.utils.ch_mapping}}
\item\relax\sphinxstyleindexentry{pycatch.utils.extensions}\sphinxstyleindexpageref{pycatch/utils/extensions:\detokenize{module-pycatch.utils.extensions}}
\item\relax\sphinxstyleindexentry{pycatch.utils.plot}\sphinxstyleindexpageref{pycatch/utils/plot:\detokenize{module-pycatch.utils.plot}}
\end{sphinxtheindex}

\renewcommand{\indexname}{Index}
\printindex
\end{document}